%%%%%%%%%%%%%%%%%%%%%%%%%%%%%%%%%%%%%%%%%%%%%%%
%%% Template for lab reports used at STIMA
%%%%%%%%%%%%%%%%%%%%%%%%%%%%%%%%%%%%%%%%%%%%%%%

%%%%%%%%%%%%%%%%%%%%%%%%%%%%%% Sets the document class for the document Openany
% is added to remove the book style of starting every new chapter on an odd page
% (not needed for reports)
\documentclass[11pt,english, openright, oneside]{book}

%%%%%%%%%%%%%%%%%%%%%%%%%%%%%% Loading packages that alter the style
\usepackage[]{graphicx}
\usepackage[]{color}
\usepackage{alltt}
\usepackage[T1]{fontenc}
\usepackage[utf8]{inputenc}
\usepackage{subcaption}
\usepackage{listings}
\usepackage{afterpage}
\usepackage{enumitem}
\newcommand\blankpage{%
    \null
    \thispagestyle{empty}%
    \newpage}

\usepackage[english, portuguese]{babel}

\usepackage[colorlinks = true, linkcolor = blue, urlcolor  = blue, citecolor =
            blue, anchorcolor = blue]{hyperref}

\newcommand{\MYhref}[3][blue]{\href{#2}{\color{#1}{#3}}}%

\renewcommand{\lstlistingname}{Anexos de Código}
\renewcommand{\lstlistlistingname}{Lista de \lstlistingname}

\setcounter{secnumdepth}{3}
\setcounter{tocdepth}{3}
\setlength{\parskip}{\smallskipamount}
\setlength{\parindent}{0pt}

\usepackage{listings}
\usepackage{color}

\definecolor{dkgreen}{rgb}{0,0.6,0}
\definecolor{gray}{rgb}{0.5,0.5,0.5}
\definecolor{mauve}{rgb}{0.58,0,0.82}
% Define some colors - Do professor
\definecolor{ListingColorKeyWord}{rgb}{0, 0.5, 0}
\definecolor{ListingColorComment}{rgb}{0.0, 0.0, 0.6}
\definecolor{ListingColorIdentifier}{rgb}{0.5, 0.12, 0.10}
\definecolor{ListingColorEmphasize}{rgb}{0, 1, 1}

\definecolor{ListingColorBreakLine}{rgb}{0.5, 0.12, 0.10}

\lstset{frame=tb, language=Java, aboveskip=3mm, belowskip=3mm,
  showstringspaces=false, columns=flexible, basicstyle={\small\ttfamily},
  numbers=none, numberstyle=\tiny\color{gray}, keywordstyle=\color{blue},
  commentstyle=\color{dkgreen}, stringstyle=\color{mauve}, breaklines=true,
  breakatwhitespace=true, tabsize=3 }

\lstset{ language=Python, keywordstyle={\color{ListingColorKeyWord}\bfseries},
	commentstyle=\color{ListingColorComment},
	identifierstyle=\color{ListingColorIdentifier}, basicstyle=\ttfamily,
	frame=single, showstringspaces=false, numbers=left, tabsize=2,
	breaklines=true,
	postbreak=\mbox{\textcolor{ListingColorBreakLine}{$\hookrightarrow$}}, }


\lstdefinelanguage{HTML5}{ language=html, sensitive=true, alsoletter={<>=-},
        otherkeywords={
        % HTML tags
        <html>, <head>, <title>, </title>, <meta, />, </head>, <body>, <canvas,
        \/canvas>, <script>, </script>, </body>, </html>, <!, html>, <style>,
        </style>, >< },  
        ndkeywords={
        % General
        =,
        % HTML attributes
        charset=, id=, width=, height=,
        % CSS properties
        border:, transform:, -moz-transform:, transition-duration:,
        transition-property:, transition-timing-function: },  
        morecomment=[s]{<!--}{-->}, tag=[s] }

\lstdefinelanguage{JavaScript}{ morekeywords={typeof, new, true, false, catch,
  function, return, null, catch, switch, var, if, in, while, do, else, case,
  break}, morecomment=[s]{/*}{*/}, morecomment=[l]//, morestring=[b]",
  morestring=[b]' }

% Set page margins
\usepackage[top=100pt,bottom=100pt,left=68pt,right=66pt]{geometry}

% Package used for placeholder text
\usepackage{lipsum}

% Prevents LaTeX from filling out a page to the bottom
\raggedbottom

% Adding both languages \usepackage[english, italian, portuguese]{babel}



\lstset{ language=python, %% Troque para PHP, C, Java, etc... bash é o padrão
    basicstyle=\ttfamily\small, numberstyle=\footnotesize, numbers=left,
    frame=single, tabsize=2, rulecolor=\color{black!30}, title=\lstname,
    escapeinside={\%*}{*)}, breaklines=true, breakatwhitespace=true,
    framextopmargin=2pt, framexbottommargin=2pt, inputencoding=utf8,
    extendedchars=true, literate={á}{{\'a}}1 {ã}{{\~a}}1 {é}{{\'e}}1
    {ç}{{\c{c}}}1, }


% All page numbers positioned at the bottom of the page
\usepackage{fancyhdr}
\fancyhf{} % clear all header and footers
\fancyfoot[C]{\thepage}
\renewcommand{\headrulewidth}{0pt} % remove the header rule
\pagestyle{fancy}
\renewcommand{\headrulewidth}{0.4pt}

% Changes the style of chapter headings
\usepackage{titlesec}
\titleformat{\chapter}
   {\normalfont\LARGE\bfseries}{\thechapter.}{1em}{}
% Change distance between chapter header and text
\titlespacing{\chapter}{0pt}{5pt}{2\baselineskip}

% Adds table captions above the table per default
\usepackage{float}
\floatstyle{plaintop}
\restylefloat{table}

% Adds space between caption and table
\usepackage[tableposition=top]{caption}

% Adds hyperlinks to references and ToC
\usepackage{hyperref}
\hypersetup{hidelinks,linkcolor = black} % Changes the link color to black and
% hides the hideous red border that usually is created

% If multiple images are to be added, a folder (path) with all the images can be
% added here 
\graphicspath{ {Figures/} }

% Separates the first part of the report/thesis in Roman numerals
\frontmatter

\usepackage[nottoc,numbib]{tocbibind}

\usepackage{listings}
\usepackage{color}

\usepackage{titlesec}
\titleformat{\chapter}[display]
  {\centering \normalsize \huge  \color{black}}{\thechapter}{10pt}{}



\definecolor{dkgreen}{rgb}{0,0.6,0}
\definecolor{gray}{rgb}{0.5,0.5,0.5}
\definecolor{mauve}{rgb}{0.58,0,0.82}

\lstset{frame=tb, language=Java, aboveskip=3mm, belowskip=3mm,
  showstringspaces=false, columns=flexible, basicstyle={\small\ttfamily},
  numbers=none, numberstyle=\tiny\color{gray}, keywordstyle=\color{blue},
  commentstyle=\color{dkgreen}, stringstyle=\color{mauve}, breaklines=true,
  breakatwhitespace=true, tabsize=3 }

%%%%%%%%%%%%%%%%%%%%%%%%%%%%%% Starts the document
\begin{document}

%%% Selects the language to be used for the first couple of pages
\selectlanguage{portuguese}


\renewcommand{\contentsname}{Índice}

%%%%% Adds the title page
\begin{titlepage}
	\clearpage\thispagestyle{empty}
	\centering
	\vspace{1cm}

	% Titles Information about the University
	{\Large \textbf{Redes de Internet}\par} {\Large Departamento de Engenharia
	Eletrónica e Telecomunicações e de Computadores \par} {\Large Instituto
	Superior de Engenharia de Lisboa \par}
		
	\vspace{0.5cm}
    
    \centering \includegraphics[scale=0.7]{imagens/ISEL.png}

	\vspace{1cm}
	
	{\Huge \textbf{Trabalho nº 3 (BGP for internet Configuration)}} \\
	\vspace{1cm}

        {\Large Licenciatura em Engenharia Informática e Multimédia}
        
	\vspace{0.5cm}
	
	
	
	
	\begin{center}
	{\normalsize Docente: \par Prof. Nuno Cruz \\
	
        \vspace{0.5cm}
              
        Alunos (Grupo 05):
        \par
        Alexandre Ferreira nº47485 
        \par 
        João Gonçalves nº47507
        \par
        Filipe Mendes nº48628
        
        \vspace{0.5cm} 
        Turma 52D
                
        \vspace{1cm}
        {\normalsize \today \par}
	             
	             
	             
	             \par}
	\end{center}
		
	% Set the date
	
	
	\pagebreak

\end{titlepage}

\tableofcontents
% Adds a table of contents
\pagebreak
\newpage


% Adds list of figures
\begingroup
\let\clearpage\relax
\pagebreak
\listoffigures
\endgroup

\newpage

% Adds list of tables
\begingroup
\let\clearpage\relax
\pagebreak
\listoftables
\endgroup

\newpage

\mainmatter
\chapter{Introdução}
\vspace{0.2cm}

\chapter{Enquadramento Teórico}




    
\begin{table}[H]
\centering
\begin{tabular}{|c|c|c|c|}
\hline
\textbf{Router} & \textbf{Interface} & \textbf{Private IP Address} & \textbf{Public IP Address} \\ \hline
R1 & Loopback0 & 10.1.1.1 /32 & \\ \hline
R1 & Loopback1 & & 47.73.239.11 /32 \\ \hline
R1 & G1/0 & 10.1.2.1 /30 & \\ \hline
R1 & G2/0 & 10.1.3.1 /30 & \\ \hline
R1 & G4/0 &  & 47.73.240.1 /30 \\ \hline
R2 & Loopback0 & 10.2.2.2 /32 & \\ \hline
R2 & Loopback1 & & 47.73.239.22 /32 \\ \hline
R2 & G1/0 & 10.0.1.1 & \\ \hline
R2 & G2/0 & 10.0.4.1 & \\ \hline
R2 & f0/0 & ?.?.?.? & ?.?.?.? \\ \hline
R3 & Loopback0 & 10.3.3.3 /32 & \\ \hline
R3 & Loopback1 & & 47.73.239.33 /32  \\ \hline
R3 & G1/0 & 10.3.4.1 /30 & \\ \hline
R3 & G2/0 & 10.1.3.1 /30 & \\ \hline
R3 & G3/0 & 10.3.5.1 /30 & \\ \hline
R3 & G4/0 & & 47.73.240.5 /30 \\ \hline
R4 & Loopback0 & 10.4.4.4 /32 & \\ \hline
R4 & Loopback1 & & 47.73.239.44 /32 \\ \hline
R4 & G1/0 & 10.3.4.2 /30 & \\ \hline
R4 & G2/0 & 10.2.4.2 /30 & \\ \hline
R4 & G3/0 & 10.4.6.1 /30 & \\ \hline
R5 & Loopback0 & 10.5.5.5 /32 & \\ \hline
R5 & Loopback1 & & 47.73.239.55 /32 \\ \hline
R5 & G1/0 & & 63.112.0.2 /30 \\ \hline
R5 & G3/0 & 10.3.5.2 /30 & \\ \hline
R6 & Loopback0 & 10.6.6.6 /32 & \\ \hline
R6 & Loopback1 & & 47.73.239.66 /32 \\ \hline
R6 & f0/0 & & 47.73.240.17 /30 \\ \hline
R6 & G1/0 & & 47.73.240.13 /30 \\ \hline
R6 & G2/0 & & 47.73.240.21 /30\\ \hline
R6 & G3/0 & 10.4.6.2 /30& \\ \hline
R7 & Loopback0 & 10.7.7.7 /32 & \\ \hline
R7 & Loopback1 & & 129.41.46.77 /32 \\ \hline
R7 & f0/0 & & 129.41.46.9 /30 \\ \hline
R7 & G1/0 & 10.7.8.1 /30 & \\ \hline
R7 & G4/0 & & 47.73.240.6 /30 \\ \hline
R7 & G5/0 & & 63.112.0.6 /30 \\ \hline
\end{tabular}
\caption{Endereços IP das interfaces do Router 1}
\label{tab:ip1}
\end{table}


\begin{table}[H]
\centering
\begin{tabular}{|c|c|c|c|}
\hline
\textbf{Router} & \textbf{Interface} & \textbf{IP Address} & \textbf{Subnet Mask} \\ \hline
R8 & Loopback0 & 10.8.8.8 /32 & \\ \hline
R8 & Loopback1 & & 129.41.46.88 /32 \\ \hline
R8 & f0/0 & & 129.41.46.5 /30 \\ \hline
R8 & G1/0 & 10.7.6.2 /30 & \\ \hline
R8 & G4/0 & & 47.73.240.2 /30 \\ \hline
R8 & G5/0 & & 129.41.46.2 /30 \\ \hline
R9 & Loopback0 & 10.9.9.9 /32 & \\ \hline
R9 & Loopback1 & & 129.41.47.99 /32 \\ \hline
R9 & f0/0 & & 129.41.47.10 /30 \\ \hline
R9 & f1/0 & & 129.41.47.6 /30\\ \hline
R10 & Loopback0 & 10.10.10.10 /32 & \\ \hline
R10 & Loopback1 & & 45.87.162.110 /32 \\ \hline
R10 & G1/0 & 10.10.11.1 /30 & \\ \hline
R10 & G2/0 & & 210.176.129.2 /30 \\ \hline
R10 & G3/0 & 10.10.12.1 /30 & \\ \hline
R10 & G4/0 & & 45.88.20.1 /30 \\ \hline
R11 & Loopback0 & 10.11.11.11 /32 & \\ \hline
R11 & Loopback1 & & 45.87.162.111 /32 \\ \hline
R11 & f0/0 & & 45.88.20.5 /30 \\ \hline
R11 & G1/0 & 10.10.11.2 /30 & \\ \hline
R11 & G2/0 & 10.11.12.1 /30 & \\ \hline
R12 & Loopback0 & 10.12.12.12 /32 & \\ \hline
R12 & Loopback1 & & 45.87.162.112 /32 \\ \hline
R12 & f0/0 & & 45.87.162.2 /30 \\ \hline
R12 & G2/0 & 10.11.12.2 /30 & \\ \hline
R12 & G3/0 & 10.10.12.2 /30 & \\ \hline
R12 & G5/0 & & 47.73.240.18 /30 \\ \hline
R13 & Loopback0 & 10.13.13.13 /32 & \\ \hline
R13 & Loopback1 & & 157.23.228.113 /32 \\ \hline
R13 & f0/0 & & 45.88.20.6 /30 \\ \hline
R13 & G1/0 & & 157.23.228.2 /30 \\ \hline
R14 & Loopback0 & 10.14.14.14 /32 & \\ \hline
R14 & Loopback1 & & 63.96.0.114 /32 \\ \hline
R14 & f0/0 & & 63.96.0.2 /30 \\ \hline
R14 & G1/0 & & 63.112.0.1 /30 \\ \hline
R14 & G3/0 & & 63.112.0.9 /30 \\ \hline
R14 & G5/0 & & 63.112.0.5 /30 \\ \hline
R15 & Loopback0 & 10.15.15.15 /32 & \\ \hline
R15 & Loopback1 & & 210.176.128.115 /32 \\ \hline
R15 & f0/0 & & 210.176.128.2 /30 \\ \hline
R15 & G1/0 & & 47.73.240.14 /30 \\ \hline
R15 & G2/0 & & 210.176.129.1 /30 \\ \hline
R15 & G3/0 & & 63.112.0.10 /30 \\ \hline
R15 & G4/0 &  & 210.176.129.5 /30 \\ \hline
R16 & Loopback0 & 10.16.16.16 /32 & \\ \hline
R16 & G2/0 & & 47.73.250.22 /30 \\ \hline
R16 & G4/0 & & 45.88.20.2 /30 \\ \hline
Badguy & Loopback0 & 10.66.66.66 /32 & \\ \hline
Badguy & G4/0 & & 210.176.129.6 /30 \\ \hline
Server1 & e0 & & 129.41.46.1 /30 \\ \hline
Server2 & e0 & & 47.73.239.1 /30 \\ \hline
Server3 & e0 & & 63.96.0.1 /30 \\ \hline
Server4 & e0 & & 210.176.128.1 /30 \\ \hline
Server5 & e0 & & 45.87.162.1 /30 \\ \hline
Server6 & e0 & & 157.23.228.1 /30 \\ \hline
\end{tabular}
\caption{Endereços IP das interfaces do Router 1}
\label{tab:ip1}
\end{table}

\pagebreak

\chapter{Conclusões}
\vspace{0.2cm}


\pagebreak

\begin{thebibliography}{4} % 100 is a random guess of the total number of
  %references
  \bibitem{Slides} Documentos de apoio da UC e material fornecido pelo docente
\end{thebibliography}

\mainmatter
\end{document}