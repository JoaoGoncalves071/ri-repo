%%%%%%%%%%%%%%%%%%%%%%%%%%%%%%%%%%%%%%%%%%%%%%%
%%% Template for lab reports used at STIMA
%%%%%%%%%%%%%%%%%%%%%%%%%%%%%%%%%%%%%%%%%%%%%%%

%%%%%%%%%%%%%%%%%%%%%%%%%%%%%% Sets the document class for the document Openany
% is added to remove the book style of starting every new chapter on an odd page
% (not needed for reports)
\documentclass[11pt,english, openright, oneside]{book}

%%%%%%%%%%%%%%%%%%%%%%%%%%%%%% Loading packages that alter the style
\usepackage[]{graphicx}
\usepackage[]{color}
\usepackage{alltt}
\usepackage[T1]{fontenc}
\usepackage[utf8]{inputenc}
\usepackage{subcaption}
\usepackage{listings}
\usepackage{afterpage}
\newcommand\blankpage{%
    \null
    \thispagestyle{empty}%
    \newpage}

\usepackage[english, portuguese]{babel}

\usepackage[colorlinks = true, linkcolor = blue, urlcolor  = blue, citecolor =
            blue, anchorcolor = blue]{hyperref}

\newcommand{\MYhref}[3][blue]{\href{#2}{\color{#1}{#3}}}%

\renewcommand{\lstlistingname}{Anexos de Código}
\renewcommand{\lstlistlistingname}{Lista de \lstlistingname}

\setcounter{secnumdepth}{3}
\setcounter{tocdepth}{3}
\setlength{\parskip}{\smallskipamount}
\setlength{\parindent}{0pt}

\usepackage{listings}
\usepackage{color}

\definecolor{dkgreen}{rgb}{0,0.6,0}
\definecolor{gray}{rgb}{0.5,0.5,0.5}
\definecolor{mauve}{rgb}{0.58,0,0.82}
% Define some colors - Do professor
\definecolor{ListingColorKeyWord}{rgb}{0, 0.5, 0}
\definecolor{ListingColorComment}{rgb}{0.0, 0.0, 0.6}
\definecolor{ListingColorIdentifier}{rgb}{0.5, 0.12, 0.10}
\definecolor{ListingColorEmphasize}{rgb}{0, 1, 1}

\definecolor{ListingColorBreakLine}{rgb}{0.5, 0.12, 0.10}

\lstset{frame=tb, language=Java, aboveskip=3mm, belowskip=3mm,
  showstringspaces=false, columns=flexible, basicstyle={\small\ttfamily},
  numbers=none, numberstyle=\tiny\color{gray}, keywordstyle=\color{blue},
  commentstyle=\color{dkgreen}, stringstyle=\color{mauve}, breaklines=true,
  breakatwhitespace=true, tabsize=3 }

\lstset{ language=Python, keywordstyle={\color{ListingColorKeyWord}\bfseries},
	commentstyle=\color{ListingColorComment},
	identifierstyle=\color{ListingColorIdentifier}, basicstyle=\ttfamily,
	frame=single, showstringspaces=false, numbers=left, tabsize=2,
	breaklines=true,
	postbreak=\mbox{\textcolor{ListingColorBreakLine}{$\hookrightarrow$}}, }


\lstdefinelanguage{HTML5}{ language=html, sensitive=true, alsoletter={<>=-},
        otherkeywords={
        % HTML tags
        <html>, <head>, <title>, </title>, <meta, />, </head>, <body>, <canvas,
        \/canvas>, <script>, </script>, </body>, </html>, <!, html>, <style>,
        </style>, >< },  
        ndkeywords={
        % General
        =,
        % HTML attributes
        charset=, id=, width=, height=,
        % CSS properties
        border:, transform:, -moz-transform:, transition-duration:,
        transition-property:, transition-timing-function: },  
        morecomment=[s]{<!--}{-->}, tag=[s] }

\lstdefinelanguage{JavaScript}{ morekeywords={typeof, new, true, false, catch,
  function, return, null, catch, switch, var, if, in, while, do, else, case,
  break}, morecomment=[s]{/*}{*/}, morecomment=[l]//, morestring=[b]",
  morestring=[b]' }

% Set page margins
\usepackage[top=100pt,bottom=100pt,left=68pt,right=66pt]{geometry}

% Package used for placeholder text
\usepackage{lipsum}

% Prevents LaTeX from filling out a page to the bottom
\raggedbottom

% Adding both languages \usepackage[english, italian, portuguese]{babel}



\lstset{ language=python, %% Troque para PHP, C, Java, etc... bash é o padrão
    basicstyle=\ttfamily\small, numberstyle=\footnotesize, numbers=left,
    frame=single, tabsize=2, rulecolor=\color{black!30}, title=\lstname,
    escapeinside={\%*}{*)}, breaklines=true, breakatwhitespace=true,
    framextopmargin=2pt, framexbottommargin=2pt, inputencoding=utf8,
    extendedchars=true, literate={á}{{\'a}}1 {ã}{{\~a}}1 {é}{{\'e}}1
    {ç}{{\c{c}}}1, }


% All page numbers positioned at the bottom of the page
\usepackage{fancyhdr}
\fancyhf{} % clear all header and footers
\fancyfoot[C]{\thepage}
\renewcommand{\headrulewidth}{0pt} % remove the header rule
\pagestyle{fancy}
\renewcommand{\headrulewidth}{0.4pt}

% Changes the style of chapter headings
\usepackage{titlesec}
\titleformat{\chapter}
   {\normalfont\LARGE\bfseries}{\thechapter.}{1em}{}
% Change distance between chapter header and text
\titlespacing{\chapter}{0pt}{5pt}{2\baselineskip}

% Adds table captions above the table per default
\usepackage{float}
\floatstyle{plaintop}
\restylefloat{table}

% Adds space between caption and table
\usepackage[tableposition=top]{caption}

% Adds hyperlinks to references and ToC
\usepackage{hyperref}
\hypersetup{hidelinks,linkcolor = black} % Changes the link color to black and
% hides the hideous red border that usually is created

% If multiple images are to be added, a folder (path) with all the images can be
% added here 
\graphicspath{ {Figures/} }

% Separates the first part of the report/thesis in Roman numerals
\frontmatter

\usepackage{listings}
\usepackage{color}

\usepackage{titlesec}
\titleformat{\chapter}[display]
  {\centering \normalsize \huge  \color{black}}{\thechapter}{10pt}{}



\definecolor{dkgreen}{rgb}{0,0.6,0}
\definecolor{gray}{rgb}{0.5,0.5,0.5}
\definecolor{mauve}{rgb}{0.58,0,0.82}

\lstset{frame=tb, language=Java, aboveskip=3mm, belowskip=3mm,
  showstringspaces=false, columns=flexible, basicstyle={\small\ttfamily},
  numbers=none, numberstyle=\tiny\color{gray}, keywordstyle=\color{blue},
  commentstyle=\color{dkgreen}, stringstyle=\color{mauve}, breaklines=true,
  breakatwhitespace=true, tabsize=3 }

%%%%%%%%%%%%%%%%%%%%%%%%%%%%%% Starts the document
\begin{document}

%%% Selects the language to be used for the first couple of pages
\selectlanguage{portuguese}


\renewcommand{\contentsname}{Índice}

%%%%% Adds the title page
\begin{titlepage}
	\clearpage\thispagestyle{empty}
	\centering
	\vspace{1cm}

	% Titles Information about the University
	{\Large \textbf{Redes de Internet}\par} {\Large Departamento de Engenharia
	Eletrónica e Telecomunicações e de Computadores \par} {\Large Instituto
	Superior de Engenharia de Lisboa \par}
		
	\vspace{0.5cm}
    
    \centering \includegraphics[scale=0.7]{imagens/isel.png}

	\vspace{1cm}
	
	{\Huge \textbf{Trabalho nº 1 (VLAN/STP/RIP)}} \\
	\vspace{1cm}

        {\Large Licenciatura em Engenharia Informática e Multimédia}
        
	\vspace{1cm}
	
	
	
	
	\begin{center}
	{\normalsize Docente: \par Prof. Nuno Garcia \\
	
        \vspace{0.5cm}
              
        Alunos (Grupo 05):
        \par
        Alexandre Ferreira nº47485 
        \par 
        João Gonçalves nº47507
        \par
        Filipe Mendes nº48628
        
        \vspace{0.5cm} 
        Turma 52D
                
        \vspace{1cm}
        {\normalsize \today \par}
	             
	             
	             
	             \par}
	\end{center}
		
	% Set the date
	
	
	\pagebreak

\end{titlepage}

\tableofcontents
% Adds a table of contents
\pagebreak
\newpage


% Adds list of figures
\begingroup
\let\clearpage\relax
\pagebreak
\listoffigures
\endgroup

\newpage

% Adds list of tables
\begingroup
\let\clearpage\relax
\pagebreak
\listoftables
\endgroup

\newpage

\mainmatter
\chapter{Introdução}


\chapter{Desenvolvimento}

\section{Tarefa 1}
\vspace{0.2cm}

Implementar a rede correspondente à Empresa A e responder às questões
apresentadas (não se esqueça de justificar as suas respostas):  

\vspace{0.8cm}

\textbf{a) Use o comando: "no ip domain-lookup". Qual o objetivo deste comando?}
\vspace{0.2cm}

O comando \textbf{"no ip domain-lookup"} é utilizado para desativar a pesquisa
de DNS num router. Vai evitar atrasos quando se cometem erros de digitação ou
comandos inválidos, pois vai desativar o router de tentar resolver nomes no
domínio desconhecidos.
\vspace{0.8cm}


\textbf{b) Quais as VLAN por omissão que existem [sh vlan] antes de ser configurada qualquer VLAN em qualquer equipamento?}
\vspace{0.2cm}

Existem 5 \textit{VLANs}: Default, fddi-default, token-ring-default,
fddinet-default, trnet-default.

Como se pode ver na figura \ref{fig:1b}
\vspace{0.4cm}

\begin{figure}[htp]
    \centering
    \includegraphics[width=0.6\textwidth]{imagens/Tarefa1/1.b.png}
    \caption{Tarefa 1 b) - VLANs no router A}
    \label{fig:1b}
\end{figure}

\vspace{0.8cm}


\pagebreak
\textbf{c) Qual o formato da tags introduzidas nas tramas Ethernet nas ligações trunk?}
\vspace{0.2cm}

Uma \textit{trunk} transporta várias \textit{VLANs} e, normalmente, são usadas
para ligações entre \textit{switches}. As tramas usadas nestas ligações contêm
campos adicionais para identificar a que VLAN pertencem.

O formato das tags segue o padrão \textit{\textbf{802.1Q}}, como se pode ver na
figura \ref{fig:1c}.
\vspace{0.4cm}

\begin{figure}[htp]
    \centering
    \includegraphics[width=0.8\textwidth]{imagens/Tarefa1/1.c.png}
    \caption{Tarefa 1 c) - Formato das tags 802.1Q}
    \label{fig:1c}
\end{figure} 

\vspace{0.8cm}


\textbf{d) Qual a razão pela qual, numa LAN que utilize VLANs, numa ligação tipo Access as tramas não incluem tags?}
\vspace{0.2cm}

Enquanto que uma porta trunk é usada em switches para transportar múltiplas
VLANs, a porta access é usada para conectar dispositivos finais à rede, como
PCs. Estes dispositivos não processam as tags da VLAN, por isso, não precisam se
preocupar com a segmentação em VLANs, pois a separação das ligações é uma
responsabilidade dos switches.
\vspace{0.8cm}


\textbf{e) Qual é a tag que as tramas pertencentes à VLAN 1 transportam?}
\vspace{0.2cm}

Por padrão, a VLAN nativa é VLAN 1 e, por isso, as tramas pertencentes a esta
VLAN são enviadas sem tag (untagged). Isto significa que quando uma interface
está configurada na VLAN 1, as tramas circulam nessa interface sem identificação
específica de VLAN. 

Se um dispositivo não suportar trunking, a circulação de tramas é possível
através da sua VLAN nativa.
\vspace{0.8cm}


\textbf{f) Uma máquina quando recebe uma trama Ethernet como diferencia se esta a seguir ao campo endereço de origem inclui o campo do tipo Type/Lenght ou se inclui os campos associados a uma VLAN?}
\vspace{0.2cm}

A máquina identifica a presença de uma tag VLAN verificando se o campo logo após
o endereço de origem é 0x8100. Se for, é uma trama com VLAN e, se não for, é uma
trama não está etiquetada com VLAN.
\vspace{0.8cm}


\pagebreak
\textbf{g) Quais as possíveis consequências de passarmos os timers “Max Age”=20 sec e “Forward Delay”= 15 sec para metade desses valores?}
\vspace{0.2cm}

Ao passarmos os timers "Max Age" e "Forward Delay" para metade dos seus valores
padrão podemos, por um lado, ser afetados positivamente na medida em que essa
mudança possibilita convergência mais rápida, podendo a rede ajustar-se mais
rapidamente a mudanças na topologia, como a adição ou remoção de dispositivos.
No entanto, as consequências negativas acabam por ser maiores pois adiciona-se
um maior risco de instabilidade e uma diminuição na resiliência a falha
temporárias. Passa a ser possível uma mais rápida mudança de estados e mudanças
como alterações na topologia, o que pode levar a que apareçam erros que não
existiriam, ou que seriam rapidamente solucionados, caso a rede convergisse com
menos rapidez.
\vspace{0.8cm}


\textbf{i) Qual é a Root Bridge (RB)? Justifique.}
\label{quest:1i}
\vspace{0.2cm}

O router SW\_DW é a root bridge porque a saída do switch diz "This bridge is the
root" e o endereço MAC no campo Address é o mesmo do SW\_DW.

Como se pode ver na figura \ref{fig:1i}
\vspace{0.4cm}
\begin{figure}[h]
    \centering
    \begin{subfigure}{.47\textwidth}
        \centering
        \includegraphics[width=0.99\linewidth]{imagens/Tarefa1/1.i.png}
    \end{subfigure}%
    \begin{subfigure}{.53\textwidth}
        \centering
        \includegraphics[width=0.99\linewidth]{imagens/Tarefa1/1.i.2.png}
    \end{subfigure}
    \caption{Tarefa 1 i) - Root Bridge}
    \label{fig:1i}
\end{figure}

\newpage

\textbf{j) Por omissão qual é o tipo de Spanning-Tree (STP) ativo [sh span]?}
\vspace{0.2cm}

Por omissão, o protocolo Spanning Tree Protocol em switches é o \textbf{IEEE
802.1D}.

Como se pode ver na figura \ref{fig:1j}
\vspace{0.4cm}

\begin{figure}[H]
    \centering
    \includegraphics[width=0.6\textwidth]{imagens/Tarefa1/1.j.png}
    \caption{Tarefa 1 j) - comando sh span}
    \label{fig:1j}
\end{figure}

\vspace{0.8cm}

\textbf{k) Quantas árvores (spanning trees) existem na topologia implementada?}
\vspace{0.2cm}

Como usa o protocolo IEEE 802.1D, contêm apenas uma árvore que é criada para
cada VLAN. Ou seja, como só existe uma VLAN, só há uma árvore.

Como se pode ver na figura \ref{fig:1k}
\vspace{0.4cm}

\begin{figure}[H]
    \centering
    \includegraphics[width=0.6\textwidth]{imagens/Tarefa1/1.k.png}
    \caption{Tarefa 1 k) - Árvores}
    \label{fig:1k}
\end{figure}

\vspace{0.8cm}


\textbf{l) Para a empresa A, construa a tabela de cálculo do custo dos caminhos e de determinação de quais são as portas Root, Designated e Blocking e calcule os respetivos valores (verifique no PT quais os valores dos custos utilizados nos cálculos das spanning trees). Os resultados finais a que chegou são coerentes com os que o simulador PT apresenta?}
\vspace{0.2cm}

\begin{table}[h!]
\centering
\begin{tabular}{|c|c|c|c|c|c|c|}
 \hline
 \textbf{Porta} & \textbf{PC} & \textbf{RPC} & \textbf{RP} &  \textbf{DPC} &
 \textbf{DP} & \textbf{Block}\\
 \hline
 SW\_DC  Gi1/0/1 & 4 & - & - & 0 & X & -\\
 SW\_DC  Gi1/0/2 & 4 & - & - & 0 & X & -\\
 \hline
 \hline
 SW1\_P1  Fa0/2 & 19 & 23 & - & 4 & X & -\\
 SW1\_P1  Fa0/23 & 19 & 42 & - & 4 & X & -\\
 SW1\_P1  Fa0/24 & 18 & 41 & - & 4 & X & -\\
 SW1\_P1  Gi0/1 & 4 & 4 & X & - & - & -\\
 \hline
 \hline
 SW2\_P1  Fa0/1 & 19 & 60 & - & 4 & X & -\\
 SW2\_P1 Fa0/2 & 19 & 23 & - & 4 & - & X\\
 SW2\_P1  Gi0/1 & 4 & 4 & X & - & - & -\\
 SW2\_P1  Fa0/24 & 19 & 60 & - & 4 & X & -\\
 SW2\_P1  Fa0/23 & 19 & 41 & - & 4 & X & -\\
 \hline
 \hline
 SW1\_P2  Fa0/2 & 19 & 42 & - & 23 & - & X\\
 SW1\_P2  Fa0/18 & 19 & 23 & - & 22 & - & X\\
 SW1\_P2  Fa0/23 & 19 & 23 & - & 23 & - & X\\
 SW1\_P2  Fa0/24 & 19 & 22 & X & - & - & -\\
 \hline
 \hline
 SW2\_P2  Fa0/2 & 19 & 41 & - & 23 & X & -\\
 SW2\_P2  Fa0/1 & 19 & 23 & X & - & - & -\\
 SW2\_P2  Fa0/24 & 19 & 23 & - & 4 & - & X\\
 \hline
\end{tabular}
\caption{Spanning Tree}
\label{tab:span}
\end{table}
\vspace{0.8cm}


\textbf{m) Qual o custo do caminho mais curto até ao Router A desde o PC9?}
\vspace{0.2cm}

 O Custo do caminho mais curto do PC9 até ao Router A é 46.
 \vspace{0.1cm}
 
 \par PC9 -> SW2 Piso 2 = 19 \par SW2 Piso 2-> SW2 Piso 1 = 19 \par SW2 Piso 1->
 SW DC = 4 \par SW DC-> Router = 4
\vspace{0.8cm}


\textbf{n) Force a root bridge para ser o SW\_DC através da prioridade. Possui alguma porta bloqueada?}
\vspace{0.2cm}

Como foi respondido anteriormente na alínea i, SW\_DC é a root bridge.
\vspace{0.8cm}


\textbf{o) Na literatura sobre spanning tree encontra-se frequentemente a afirmação de que todas as portas de um root switch/bridge são portas Designated. Comente tendo em consideração o SW\_DC.}
\vspace{0.2cm}

Esta afirmação está correta de acordo com o funcionamento do STP. Sendo a root
bridge o ponto central da rede, todas as suas portas serão designated ports,
pois vão encaminhar tramas para outros switches sem bloqueios e loops.

Sendo SW\_DC a root bridge, todas as suas portas vão ser designated ports. No
entanto, na sua configuração tem duas portas a ligar ao mesmo dispositivo, por
isso, uma delas vai ser bloqueada para evitar possíveis loops.
\vspace{0.8cm}


\textbf{p) Ative o modo Per-Vlan Rapid Spanning Tree. Verifique se é necessário ativá-lo em todos os switches? }
\vspace{0.2cm}

Foi usado o comando "spanning-tree mode rapid-pvst" para mudar de protocolo no
switch. Apenas esse switch mudou, os outros continuam com o mesmo protocolo, por
isso, é preciso ativar em todos os switches.

Como se pode ver na figura \ref{fig:1p}
\vspace{0.4cm}
\begin{figure}[h]
    \centering
    \begin{subfigure}{.53\textwidth}
        \centering
        \includegraphics[width=0.99\linewidth]{imagens/Tarefa1/1.p.png}
    \end{subfigure}%
    \begin{subfigure}{.53\textwidth}
        \centering
        \includegraphics[width=0.99\linewidth]{imagens/Tarefa1/1.p.2.png}
    \end{subfigure}
    \caption{Tarefa 1 p) - Spanning Tree PVRST}
    \label{fig:1p}
\end{figure}
\vspace{0.8cm}


\textbf{q) Quantas árvores passaram a existir?}
\vspace{0.2cm}

Assim como no STP padrão, PVRST cria uma única árvore de spanning para cada
VLAN. Portanto, o número de árvores é o mesmo que o número de VLANs.
\vspace{0.8cm}

\newpage
\textbf{r) Existem duas ligações entre o sw1\_piso1 e o sw1\_piso2, uma delas bloqueada. Altere a configuração de maneira a desbloquear a ligação bloqueada e a desbloquear outra. [Opcional] Indique qual a forma de proceder de maneira que as duas ligações pudessem ser usadas em simultâneo.}
\vspace{0.2cm}

 As duas portas tem valores de custos diferentes, uma tem custo de 18 e a outra
 19. Para que seja possível a troca de estados, temos que aumentar o custo da
 porta de menor custo ou diminuir o da porta de maior custo. Para isso, usamos o
 comando "spanning-tree cost 100".
 
\vspace{0.8cm}


\textbf{s) Explique de forma detalhada a razão do sw2\_piso2 escolher o caminho por omissão em detrimento de outro possível. Realize as alterações que considerar necessárias para que o caminho preferido seja outro que não o escolhido (por omissão).}
\vspace{0.2cm}

Em cada switch será eleita uma porta Root Port, sendo esta aquela que consegue
garantir um menor custo de caminho até à Root Bridge. Entre as portas do Switch
2 do Piso 2, verifica-se que o caminho das duas interfaces (fa0/24 e fa0/1) até
à root bridge tem o mesmo custo de 23. Com isto, a porta mais pequena é
escolhida, sendo a outra bloqueada. 

Tal como aconteceu na anterior alínea, uma forma de alterar este caminho passa
por alterar o valor de custo de uma das portas. Se reduzirmos, essa é a
escolhida e se aumentarmos fica a outra. 
\vspace{0.8cm}


\textbf{t) Considere a seguinte afirmação: “Com o SW\_DC como root bridge, a substituição do Hub0 por um switch, interligado entre o sw1\_piso1 e o SW\_DC, iria melhor a conetividade entre o PC5 e o Server2 pois o caminho ficava mais curto.”. Indique, justificando, se a mesma é falsa ou verdadeira atendendo a que todas as ligações ao switch novo funcionam a 100 Mbps.}
\vspace{0.2cm}
 
Anteriormente, o PC5 passava pelo sw1\_piso1, SW\_DC, Hub0 até chegar ao
Server2. Com a subsituição do Hub0 por um switch e, fazendo a ligação com a root
bridge e sw1\_piso1 iremos ter um loop. Com isto, como as ligação do novo switch
tem um custo alto, as portas vão ser bloqueadas e, por isso, esta afirmação é
falsa, porque vai efetuar o mesmo percurso que fazia.
\vspace{0.8cm}

\pagebreak



\section{Tarefa 2}

Segmente a topologia da \textbf{Empresa A} utilizando VLAN para ficar de acordo
com as regras abaixo (poderá criar outras VLAN se necessário). Os informáticos
da empresa decidiram que cada VLAN terá um endereço IPv4 privado dentro da gama
\textbf{172.16.0.0/12}. A tabela seguinte é uma sugestão. Como resultado
pretende-se a implementação no  simulador da topologia indicada e o resultado
dos testes indicados na última alínea que comprovem que a topologia está
implementada e configurada como indicado nos passos (alíneas) seguintes:  

\begin{table}[h!]
\centering
\begin{tabular}{|c|c|c|c|c|}
 \hline
 \textbf{Nº Vlan} & \textbf{Nome} & \textbf{IP do Gateway} & \textbf{Rede} &
 \textbf{PCs}\\
 \hline
 51 & Contabilidade & 172.24.51.255 & 172.24.51.0/24 & PC7, PC9\\
 52 & Secretariado & 172.24.53.255 & 172.24.53.0/24 & PC5, PC8\\
 53 & Informática & 172.24.55.127 & 172.24.55.0/25 & PC6\\
 54 & Gestão da rede & 172.24.56.255 & 172.24.56.128/25 & Server2\\
 \hline
\end{tabular}
\caption{VLANs}
\label{tab:vlan}
\end{table}
\vspace{0.8cm}

\pagebreak


\section{Tarefa 3}

\pagebreak

\chapter{Conclusões}



\chapter{Bibliografia}



\mainmatter



\afterpage{\blankpage}

\end{document}