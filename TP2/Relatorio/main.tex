%%%%%%%%%%%%%%%%%%%%%%%%%%%%%%%%%%%%%%%%%%%%%%%
%%% Template for lab reports used at STIMA
%%%%%%%%%%%%%%%%%%%%%%%%%%%%%%%%%%%%%%%%%%%%%%%

%%%%%%%%%%%%%%%%%%%%%%%%%%%%%% Sets the document class for the document Openany
% is added to remove the book style of starting every new chapter on an odd page
% (not needed for reports)
\documentclass[11pt,english, openright, oneside]{book}

%%%%%%%%%%%%%%%%%%%%%%%%%%%%%% Loading packages that alter the style
\usepackage[]{graphicx}
\usepackage[]{color}
\usepackage{alltt}
\usepackage[T1]{fontenc}
\usepackage[utf8]{inputenc}
\usepackage{subcaption}
\usepackage{listings}
\usepackage{afterpage}
\usepackage{enumitem}
\newcommand\blankpage{%
    \null
    \thispagestyle{empty}%
    \newpage}

\usepackage[english, portuguese]{babel}

\usepackage[colorlinks = true, linkcolor = blue, urlcolor  = blue, citecolor =
            blue, anchorcolor = blue]{hyperref}

\newcommand{\MYhref}[3][blue]{\href{#2}{\color{#1}{#3}}}%

\renewcommand{\lstlistingname}{Anexos de Código}
\renewcommand{\lstlistlistingname}{Lista de \lstlistingname}

\setcounter{secnumdepth}{3}
\setcounter{tocdepth}{3}
\setlength{\parskip}{\smallskipamount}
\setlength{\parindent}{0pt}

\usepackage{listings}
\usepackage{color}

\definecolor{dkgreen}{rgb}{0,0.6,0}
\definecolor{gray}{rgb}{0.5,0.5,0.5}
\definecolor{mauve}{rgb}{0.58,0,0.82}
% Define some colors - Do professor
\definecolor{ListingColorKeyWord}{rgb}{0, 0.5, 0}
\definecolor{ListingColorComment}{rgb}{0.0, 0.0, 0.6}
\definecolor{ListingColorIdentifier}{rgb}{0.5, 0.12, 0.10}
\definecolor{ListingColorEmphasize}{rgb}{0, 1, 1}

\definecolor{ListingColorBreakLine}{rgb}{0.5, 0.12, 0.10}

\lstset{frame=tb, language=Java, aboveskip=3mm, belowskip=3mm,
  showstringspaces=false, columns=flexible, basicstyle={\small\ttfamily},
  numbers=none, numberstyle=\tiny\color{gray}, keywordstyle=\color{blue},
  commentstyle=\color{dkgreen}, stringstyle=\color{mauve}, breaklines=true,
  breakatwhitespace=true, tabsize=3 }

\lstset{ language=Python, keywordstyle={\color{ListingColorKeyWord}\bfseries},
	commentstyle=\color{ListingColorComment},
	identifierstyle=\color{ListingColorIdentifier}, basicstyle=\ttfamily,
	frame=single, showstringspaces=false, numbers=left, tabsize=2,
	breaklines=true,
	postbreak=\mbox{\textcolor{ListingColorBreakLine}{$\hookrightarrow$}}, }


\lstdefinelanguage{HTML5}{ language=html, sensitive=true, alsoletter={<>=-},
        otherkeywords={
        % HTML tags
        <html>, <head>, <title>, </title>, <meta, />, </head>, <body>, <canvas,
        \/canvas>, <script>, </script>, </body>, </html>, <!, html>, <style>,
        </style>, >< },  
        ndkeywords={
        % General
        =,
        % HTML attributes
        charset=, id=, width=, height=,
        % CSS properties
        border:, transform:, -moz-transform:, transition-duration:,
        transition-property:, transition-timing-function: },  
        morecomment=[s]{<!--}{-->}, tag=[s] }

\lstdefinelanguage{JavaScript}{ morekeywords={typeof, new, true, false, catch,
  function, return, null, catch, switch, var, if, in, while, do, else, case,
  break}, morecomment=[s]{/*}{*/}, morecomment=[l]//, morestring=[b]",
  morestring=[b]' }

% Set page margins
\usepackage[top=100pt,bottom=100pt,left=68pt,right=66pt]{geometry}

% Package used for placeholder text
\usepackage{lipsum}

% Prevents LaTeX from filling out a page to the bottom
\raggedbottom

% Adding both languages \usepackage[english, italian, portuguese]{babel}



\lstset{ language=python, %% Troque para PHP, C, Java, etc... bash é o padrão
    basicstyle=\ttfamily\small, numberstyle=\footnotesize, numbers=left,
    frame=single, tabsize=2, rulecolor=\color{black!30}, title=\lstname,
    escapeinside={\%*}{*)}, breaklines=true, breakatwhitespace=true,
    framextopmargin=2pt, framexbottommargin=2pt, inputencoding=utf8,
    extendedchars=true, literate={á}{{\'a}}1 {ã}{{\~a}}1 {é}{{\'e}}1
    {ç}{{\c{c}}}1, }


% All page numbers positioned at the bottom of the page
\usepackage{fancyhdr}
\fancyhf{} % clear all header and footers
\fancyfoot[C]{\thepage}
\renewcommand{\headrulewidth}{0pt} % remove the header rule
\pagestyle{fancy}
\renewcommand{\headrulewidth}{0.4pt}

% Changes the style of chapter headings
\usepackage{titlesec}
\titleformat{\chapter}
   {\normalfont\LARGE\bfseries}{\thechapter.}{1em}{}
% Change distance between chapter header and text
\titlespacing{\chapter}{0pt}{5pt}{2\baselineskip}

% Adds table captions above the table per default
\usepackage{float}
\floatstyle{plaintop}
\restylefloat{table}

% Adds space between caption and table
\usepackage[tableposition=top]{caption}

% Adds hyperlinks to references and ToC
\usepackage{hyperref}
\hypersetup{hidelinks,linkcolor = black} % Changes the link color to black and
% hides the hideous red border that usually is created

% If multiple images are to be added, a folder (path) with all the images can be
% added here 
\graphicspath{ {Figures/} }

% Separates the first part of the report/thesis in Roman numerals
\frontmatter

\usepackage[nottoc,numbib]{tocbibind}

\usepackage{listings}
\usepackage{color}

\usepackage{titlesec}
\titleformat{\chapter}[display]
  {\centering \normalsize \huge  \color{black}}{\thechapter}{10pt}{}



\definecolor{dkgreen}{rgb}{0,0.6,0}
\definecolor{gray}{rgb}{0.5,0.5,0.5}
\definecolor{mauve}{rgb}{0.58,0,0.82}

\lstset{frame=tb, language=Java, aboveskip=3mm, belowskip=3mm,
  showstringspaces=false, columns=flexible, basicstyle={\small\ttfamily},
  numbers=none, numberstyle=\tiny\color{gray}, keywordstyle=\color{blue},
  commentstyle=\color{dkgreen}, stringstyle=\color{mauve}, breaklines=true,
  breakatwhitespace=true, tabsize=3 }

%%%%%%%%%%%%%%%%%%%%%%%%%%%%%% Starts the document
\begin{document}

%%% Selects the language to be used for the first couple of pages
\selectlanguage{portuguese}


\renewcommand{\contentsname}{Índice}

%%%%% Adds the title page
\begin{titlepage}
	\clearpage\thispagestyle{empty}
	\centering
	\vspace{1cm}

	% Titles Information about the University
	{\Large \textbf{Redes de Internet}\par} {\Large Departamento de Engenharia
	Eletrónica e Telecomunicações e de Computadores \par} {\Large Instituto
	Superior de Engenharia de Lisboa \par}
		
	\vspace{0.5cm}
    
    \centering \includegraphics[scale=0.7]{imagens/ISEL.png}

	\vspace{1cm}
	
	{\Huge \textbf{Trabalho nº 2 (OSPF LAB: ADVANCED ROUTING CONFIGURATIONS)}} \\
	\vspace{1cm}

        {\Large Licenciatura em Engenharia Informática e Multimédia}
        
	\vspace{0.5cm}
	
	
	
	
	\begin{center}
	{\normalsize Docente: \par Prof. Nuno Cruz \\
	
        \vspace{0.5cm}
              
        Alunos (Grupo 05):
        \par
        Alexandre Ferreira nº47485 
        \par 
        João Gonçalves nº47507
        \par
        Filipe Mendes nº48628
        
        \vspace{0.5cm} 
        Turma 52D
                
        \vspace{1cm}
        {\normalsize \today \par}
	             
	             
	             
	             \par}
	\end{center}
		
	% Set the date
	
	
	\pagebreak

\end{titlepage}

\tableofcontents
% Adds a table of contents
\pagebreak
\newpage


% Adds list of figures
\begingroup
\let\clearpage\relax
\pagebreak
\listoffigures
\endgroup

\newpage

% Adds list of tables
\begingroup
\let\clearpage\relax
\pagebreak
\listoftables
\endgroup

\newpage

\mainmatter
\chapter{Introdução}
Este relatório científico, elaborado no âmbito da unidade curricular de Redes de Internet, tem como objetivo explorar e aprofundar o conhecimento sobre os protocolos RIP e OSPF, bem como sobre conectividades em redes de computadores, incluindo o uso do mecanismo de ping e as funcionalidades das interfaces de rede. O trabalho foi desenvolvido com a construção e configuração de uma rede composta por hosts, switches e routers, onde foram estabelecidas conexões entre esses dispositivos para representar cenários reais de comunicação. \par \vspace{0.2cm}

Ao longo do trabalho, realizaram-se diversas tarefas que incluíram a configuração dos protocolos de encaminhamento RIP e OSPF e a verificação das conectividades entre dispositivos. Estas atividades permitiram observar o comportamento da rede em diferentes situações e testar o funcionamento dos protocolos em ambientes simulados. O relatório descreve em detalhe as configurações efetuadas, as justificações para cada escolha e os resultados obtidos, oferecendo uma visão completa das práticas de configuração e gestão aplicadas na operação de redes de computadores. \par \vspace{0.2cm}

A topologia de rede implementada é a seguinte imagem:

\begin{figure}[H]
    \centering
    \includegraphics[width=1\textwidth]{imagens/topologia.png}
    \caption{Topologia de rede}
    \label{fig:topologia}
\end{figure}

\chapter{Enquadramento Teórico}
Sendo que ao longo do documento são abordados o RIPv2 e OSPF. De seguida, sumarizar-se-ão os temas 
sob a forma de um glossário de modo a facilitar a compreensão dos tópicos:

\begin{itemize}
  \item \textbf{OSPF}
  \vspace{0.2cm}
  \par O OSPF é um protocolo do tipo \textbf{link state}, de routing interno (\textbf{IGP - Internal Gateway Protocol}), dinâmico e "aberto", que pode ser implementado por qualquer fabricante sem o pagamento de licença, permitindo assim o seu uso generalizado. Apresenta diversas vantagens, como a inexistência de limites no número de saltos (hops), suporte ao encaminhamento \textbf{classless}, e \textbf{menor tráfego}, uma vez que as atualizações dos caminhos são enviadas com intervalos mais longos ou apenas quando ocorre uma alteração na topologia. Além disso, proporciona uma \textbf{convergência rápida}, \textbf{não cria loops}, reage prontamente às mudanças na rede, oferece um \textbf{melhor balanceamento} de carga, permite a \textbf{definição lógica de áreas}, facilitando a gestão da rede através do princípio de "dividir para reinar", possibilita a \textbf{marcação de rotas externas} e \textbf{suporta autenticação}.
  \vspace{0.2cm}

  Cada router \textbf{constrói um "mapa" da topologia da sua área}, trocando mensagens de \textbf{Link State Update} entre si para anunciar as ligações que possuem. Sempre que possível, utilizam \textbf{multicast} para essa comunicação. Cada router, então, calcula o \textbf{caminho mais curto} para todos os outros routers da área, utilizando o algoritmo de \textbf{Dijkstra}. A \textbf{tabela de routing} inclui a informação resultante deste cálculo, assim como a informação proveniente dos routers que fazem fronteira com outras áreas. Para manter a integridade da rede, os routers enviam constantemente mensagens \textbf{Hello} para verificar se os outros routers estão ativos. Caso um router não responda, os restantes routers da área são notificados e recalculam os melhores caminhos.
  \vspace{0.2cm}

  No contexto do OSPF, um grupo de routers que troca informações de encaminhamento entre si é denominado \textbf{Sistema Autónomo (AS)}. Este é constituído por um conjunto de redes que pode ser subdividido em várias \textbf{áreas} menores. A divisão em áreas traz vantagens, como ocultar a topologia de cada área das outras, isolar a eventual instabilidade de uma área das restantes e permitir que os routers necessitem de menos memória, dado que, sendo as áreas menores, o número de routers e redes em cada uma é reduzido, com as rotas para as outras \textbf{áreas a serem sumarizadas}.
  
  \newpage
  Adicionalmente, é importante mencionar os vários tipos de routers no OSPF:
  \begin{itemize}
      \item \textbf{Internal Router} em ligações apenas a routers da mesma área.
      \item \textbf{Area Border Router (ABR)} tem ligações a routers de outra área 0, sendo o responsável pela troca de informações de routing entre áreas. Cada ABR numa área sumariza para a área o custo para todas as redes externas à área. Depois de ser calculada a árvore SPF 
      para a área, os caminhos para os destinos inter-área (exteriores à área) são calculados examinando os sumários dos ABR. 
      \item \textbf{Autonomous System Border Router (ASBR)} tem ligações a routers de outros Sistemas 
      Autónomos. Também pode executar outros protocolos de routing (IGP ou EGP - RIP, 
      EIGRP, BGP). 
      \item \textbf{Backbone Router} tem pelo menos uma interface que executa o OSPF na área 0.
    \end{itemize}
    \vspace{0.2cm}

  \item \textbf{RIPv2}
  \par O protocolo \textbf{RIP (Routing Information Protocol)} fornece um mecanismo de troca de mensagens contendo informações sobre rotas, de modo a manter as tabelas de encaminhamento de cada router atualizadas. As informações trocadas mais importantes incluem:
  \begin{itemize}
      \item O \textbf{endereço} de cada rede ou máquina: Identifica os destinos na rede.
      \item A \textbf{distância em hops (saltos)} do router para a rede ou máquina: Por exemplo, 1 hop indica entrega direta e 2 hops significa que a mensagem passa por um único router.
      \item O primeiro salto para a rota: Este é o local para onde os datagramas devem ser enviados para alcançar a rede ou a máquina de destino (apenas no RIPv2).
    \end{itemize}
    \vspace{0.2cm}

    O RIP apresenta duas versões, sendo que neste trabalho será utilizada apenas a segunda versão, o RIPv2. A convergência é um dos principais problemas associados ao RIP, mas o RIPv2 implementa vários mecanismos para mitigar este problema:

    \begin{itemize}
      \item \textbf{Split Horizon Update}: Evita enviar informações sobre rotas de volta pela mesma interface de onde foram recebidas.
      \item \textbf{Triggered Updates}: Permite que os routers enviem atualizações imediatamente após uma alteração na tabela de rotas, em vez de esperar pelo ciclo de atualização normal.
      \item \textbf{Split Horizon with Poisoned Reverse}: Combina o conceito de split horizon com a adição de uma rota "venenosa" para indicar que uma rota não é mais válida.
      \item \textbf{Hold Down}: Impede que alterações rápidas em rotas sejam consideradas, estabilizando a tabela de rotas durante um período de instabilidade.
      \end{itemize}
      \vspace{0.2cm}

      Além destes mecanismos, o RIPv2 também oferece funcionalidades de autenticação, como a autenticação por \textbf{password simples}, definida pelo administrador da rede, que autentica o router emissor perante os receptores. Também utiliza o método \textbf{MD5}, no qual as mensagens enviadas pelos routers incluem na primeira entrada de 20 bytes um valor (message digest) que serve para autenticar o router emissor perante os receptores.
\end{itemize}


\chapter{Desenvolvimento}

\section{Tarefa 1 - INITIAL ROUTER CONFIGURATION}
\vspace{0.2cm}

Nesta tarefa, iremos realizar uma configuração inicial abrangente de vários routers, que incluirá a definição de hostnames para facilitar a identificação na rede, a desativação da pesquisa DNS para evitar atrasos indesejados durante a introdução de comandos, a configuração das interfaces de rede com endereços IP apropriados conforme a tabela de endereçamento fornecida e, por fim, a verificação da conectividade básica entre os routers para garantir que a comunicação entre eles está a funcionar corretamente.

\vspace{0.2cm}

Foram fornecidas as seguintes tabelas:

\begin{table}[H]
\centering
\begin{tabular}{|c|c|}
\hline
\textbf{Loopbacks} & \textbf{IP address} \\ \hline
R1 & 1.1.1.1/32 \\ \hline
R2 & 2.2.2.2/32 \\ \hline
R3 & 3.3.3.3/32 \\ \hline
R4 & 4.4.4.4/32 \\ \hline
R5 & 5.5.5.5/32 \\ \hline
R6 & 6.6.6.6/32 \\ \hline
R7 & 7.7.7.7/32 \\ \hline
R8 & 8.8.8.8/32 \\ \hline
R9 & 9.9.9.9/32 \\ \hline
R10 & 10.10.10.10/32 \\ \hline
\end{tabular}
\caption{Endereços IP das Loopbacks dos Routers}
\label{tab:loopbacks}
\end{table}
\vspace{0.2cm}

\begin{table}[H]
\centering
\begin{tabular}{|c|c|c|}
\hline
\textbf{Segments} & \textbf{Net IP address} & \textbf{Mask} \\ \hline
S1 & 10.7.5.0 & /30 \\ \hline
S2 & 10.5.6.0 & /29 \\ \hline
S3 & 10.1.3.0 & /30 \\ \hline
S4 & 10.1.2.0 & /30 \\ \hline
S5 & 10.3.4.0 & /30 \\ \hline
S6 & 10.2.4.0 & /30 \\ \hline
S7 & 10.2.8.0 & /29 \\ \hline
S8 & 10.9.10.0 & /30 \\ \hline
S9 & 10.10.55.0 & /30 \\ \hline
S10 & 10.1.33.0 & /30 \\ \hline
S11 & 10.7.11.0 & /30 \\ \hline
\end{tabular}
\caption{Endereços IP das Sub-redes dos Routers}
\label{tab:subredes}
\end{table}
\vspace{0.2cm}

\subsection{HOSTNAME, SECURITY CONFIGURATION AND SAVING SETTINGS}
\vspace{0.2cm}

Nesta secção vamos:

\begin{enumerate}
  \item \textbf{Definir o hostname} de acordo com a topologia de cada router, facilitando a identificação na rede. Utilizando o comando \textit{hostname} seguido do nome pretendido, é possível alterar o nome do router.
  \item \textbf{Desativar a pesquisa DNS} para evitar que erros de escrita provoquem atrasos durante a utilização dos comandos. Para tal, é necessário utilizar o comando \textit{no ip domain-lookup}.
  \item \textbf{Configurar uma palavra-passe para o modo EXEC privilegiado}, assegurando que apenas utilizadores autorizados possam fazer alterações na configuração do dispositivo. Para tal, é necessário utilizar o comando \textit{enable secret} seguido da palavra-passe pretendida.
  \item \textbf{Configurar a linha de consola} para melhorar a segurança e o acesso ao router. Para tal, é necessário utilizar o comando \textit{line console 0} seguido de \textit{password} e \textit{login}.
  \item \textbf{Configurar as linhas VTY} para acesso remoto, definindo as autenticações necessárias para conexões seguras. Para tal, é necessário utilizar o comando \textit{line vty 0 4} seguido de \textit{password} e \textit{login}.
  \item \textbf{Guardar a configuração} realizada, garantindo que todas as alterações sejam mantidas após um eventual reinício do router. Para tal, é necessário utilizar o comando \textit{write memory}.
\end{enumerate}
\vspace{0.2cm}

No entanto, uma vez que estamos a utilizar um laboratório virtual e para sermos mais eficientes, podemos ignorar a configuração de passwords por agora e ir diretamente para o ambiente privilegiado. Para tal, utilizamos o comandos \textit{no enable secret}, \textit{line con 0} e \textit{exec-timeout 0 0}.

\vspace{0.2cm}
Tal como se pode observar na figura \ref{fig:config1}, as configurações foram realizadas com sucesso no router 1.
\vspace{0.2cm}

\begin{figure}[H]
    \centering
    \includegraphics[width=0.8\textwidth]{imagens/Tarefa1/1.init_conf.png}
    \caption{Configuração do Router 1}
    \label{fig:config1}
\end{figure}

\newpage
\subsection{INTERFACE CONFIGURATION}
\vspace{0.2cm}

De acordo com a tabela de endereçamento fornecida, vamos proceder com a configuração das interfaces dos routers. Primeiro temos que atribuir um endereço IP a cada interface, utilizando como se pode observar na seguinte tabela:

\begin{table}[H]
\centering
\begin{tabular}{|c|c|c|c|}
\hline
\textbf{Router} & \textbf{Interface} & \textbf{IP Address} & \textbf{Subnet Mask} \\ \hline
R1 & Loopback0 & 1.1.1.1 & 255.255.255.255 \\ \hline
R1 & G0/0 & 10.1.2.1 & 255.255.255.252 \\ \hline
R1 & G1/0 & 10.1.3.1 & 255.255.255.252 \\ \hline
R1 & G2/0 & 10.5.6.1 & 255.255.255.248 \\ \hline
R1 & G3/0 & 10.1.33.1 & 255.255.255.252 \\ \hline
R2 & Loopback0 & 2.2.2.2 & 255.255.255.255 \\ \hline
R2 & G0/0 & 10.1.2.2 & 255.255.255.252 \\ \hline
R2 & G1/0 & 10.2.4.1 & 255.255.255.252 \\ \hline
R2 & G2/0 & 10.2.8.1 & 255.255.255.248 \\ \hline
R3 & Loopback0 & 3.3.3.3 & 255.255.255.255 \\ \hline
R3 & G0/0 & 10.3.4.1 & 255.255.255.252 \\ \hline
R3 & G1/0 & 10.1.3.2 & 255.255.255.252 \\ \hline
R3 & G2/0 & 10.5.6.2 & 255.255.255.248 \\ \hline
R4 & Loopback0 & 4.4.4.4 & 255.255.255.255 \\ \hline
R4 & G0/0 & 10.2.4.2 & 255.255.255.252 \\ \hline
R4 & G1/0 & 10.3.4.2 & 255.255.255.252 \\ \hline
R5 & Loopback0 & 5.5.5.5 & 255.255.255.255 \\ \hline
R5 & G0/0 & 10.5.6.3 & 255.255.255.248 \\ \hline
R5 & G1/0 & 10.7.5.1 & 255.255.255.252 \\ \hline
R6 & Loopback0 & 6.6.6.6 & 255.255.255.255 \\ \hline
R6 & G0/0 & 10.5.6.4 & 255.255.255.248 \\ \hline
R7 & Loopback0 & 7.7.7.7 & 255.255.255.255 \\ \hline
R7 & G0/0 & 10.7.5.2 & 255.255.255.252 \\ \hline
R7 & G4/0 & 10.7.11.1 & 255.255.255.252 \\ \hline
R8 & Loopback0 & 8.8.8.8 & 255.255.255.255 \\ \hline
R8 & G0/0 & 10.2.8.2 & 255.255.255.248 \\ \hline
R9 & Loopback0 & 9.9.9.9 & 255.255.255.255 \\ \hline
R9 & G0/0 & 10.2.8.3 & 255.255.255.248 \\ \hline
R9 & G1/0 & 10.9.10.1 & 255.255.255.252 \\ \hline
R10 & Loopback0 & 10.10.10.10 & 255.255.255.255 \\ \hline
R10 & G0/0 & 10.9.10.2 & 255.255.255.252 \\ \hline
R10 & G1/0 & 10.10.55.1 & 255.255.255.252 \\ \hline
\end{tabular}
\caption{Endereços IP das interfaces do Router 1}
\label{tab:ip1}
\end{table}

\newpage
De seguida, vamos proceder com a configuração das interfaces dos routers, utilizando o comando \textit{interface} seguido do nome da interface, \textit{ip address} seguido do endereço IP e da máscara de sub-rede, e \textit{no shutdown} para ativar a interface. Para os routers 1 e 2, as configurações foram realizadas com sucesso, como se pode observar na figura \ref{fig:config2}.
\vspace{0.2cm}

\begin{figure}[h]
  \centering
  \begin{subfigure}{.6\textwidth}
      \centering
      \includegraphics[width=1\linewidth]{imagens/Tarefa1/2.int_conf_R1.png}
  \end{subfigure}%
  \begin{subfigure}{.4\textwidth}
      \centering
      \includegraphics[width=1\linewidth]{imagens/Tarefa1/2.int_conf_R2.png}
  \end{subfigure}
  \caption{Configuração das interfaces dos Routers 1 e 2}
  \label{fig:config2}
\end{figure}
\vspace{0.2cm}

\subsection{VERIFY BASIC CONNECTIVITY}
\vspace{0.2cm}

Primeiramente, vamos utilizar o comando \textit{show ip interface brief} para verificar o estado das interfaces dos routers. Como se pode observar na figura \ref{fig:config3}, as interfaces do router 1 estão ativas e com os endereços IP configurados corretamente.
\vspace{0.2cm}

\begin{figure}[H]
    \centering
    \includegraphics[width=1\textwidth]{imagens/Tarefa1/4.troubleshooting.png}
    \caption{Verificar o estado das interfaces do Router 1}
    \label{fig:config3}
\end{figure}
\vspace{0.2cm}

\newpage
Por fim, vamos verificar a conectividade básica entre os routers, utilizando o comando \textit{ping} seguido do endereço IP do router de destino. Para tal, é necessário verificar se o router de origem consegue enviar pacotes ICMP para o router de destino e se este consegue responder aos mesmos. Como se pode observar na figura \ref{fig:config4}, a conectividade foi verificada com sucesso entre os routers 1 e 2 e os routers 5 e 3.

\begin{figure}[h]
  \centering
  \begin{subfigure}{.558\textwidth}
      \centering
      \includegraphics[width=1\linewidth]{imagens/Tarefa1/3.ping_R1_R2.png}
  \end{subfigure}%
  \begin{subfigure}{.48\textwidth}
      \centering
      \includegraphics[width=1\linewidth]{imagens/Tarefa1/3.ping_R5_R3.png}
  \end{subfigure}
  \caption{Verificar a conectividade básica entre os Routers 1 e 2 e os Routers 5 e 3} 
  \label{fig:config4}
\end{figure}
\vspace{0.2cm}

\pagebreak


\section{Tarefa 2 - BASIC OSPF CONFIGURATION (AREA 0 AND AREA 3)}
\vspace{0.2cm}
Nesta tarefa, iremos aprofundar a configuração básica do OSPF (Open Shortest Path First), um protocolo de routing dinâmico amplamente utilizado em redes de grande escala. Esta tarefa terá como foco a implementação do OSPF nas áreas 0 e 3, começando pela configuração da área backbone, que é fundamental para a interconexão entre diferentes áreas de um Sistema Autónomo. Além disso, abordaremos a autenticação do OSPF, um aspecto crucial para garantir a segurança das comunicações entre routers. Durante este processo, também configuraremos os tipos de rede do OSPF, manipularemos os identificadores dos routers (router IDs) e controlaremos a eleição do Designated Router (DR) e do Backup Designated Router (BDR). Estes passos visam não apenas solidificar o entendimento sobre o funcionamento do OSPF, mas também preparar a rede para uma operação mais segura e eficiente.
\vspace{0.2cm}

\subsection{CONFIGURING OSPF IN AREA 0}
\vspace{0.2cm}

Para cada router da área 0, vamos configurar com os seguintes passos:
\begin{enumerate}
  \item \textbf{Ativar o OSPF} no router, utilizando o comando \textit{router ospf} seguido do número do processo OSPF.
  \item \textbf{Definir o router ID} do router, utilizando o comando \textit{router-id} seguido do endereço IP da loopback.
  \item \textbf{Configurar as interfaces} do router para o OSPF, utilizando o comando \textit{network} seguido do endereço IP da interface, da wildcard mask e da área OSPF.
\end{enumerate}
\vspace{0.2cm}

Para o router 1, as configurações foram realizadas com sucesso, como se pode observar na figura \ref{fig:config5}.
\vspace{0.2cm}

\begin{figure}[H]
    \centering
    \includegraphics[width=0.92\textwidth]{imagens/Tarefa2/5.conf_ospf_area0.png}
    \caption{Configuração do OSPF no Router 1}
    \label{fig:config5}
\end{figure}
\vspace{0.2cm}

\subsection{CONFIGURING OSPF IN AREA 3}
\vspace{0.2cm}

Para cada router da área 3, vamos configurar com os seguintes passos:
\vspace{0.2cm}

\begin{enumerate}
  \item \textbf{Ativar o OSPF} no router, utilizando o comando \textit{router ospf}
  \item \textbf{Definir o router ID} do router, utilizando o comando \textit{router-id} seguido do endereço IP da loopback.
  \item \textbf{Configurar as interfaces} do router para o OSPF, utilizando o comando \textit{network} seguido do endereço IP da interface, da wildcard mask e da área OSPF.
\end{enumerate}
\vspace{0.2cm}

Para o router 5, as configurações foram realizadas com sucesso, como se pode observar na figura \ref{fig:config6}.
\vspace{0.2cm}

\begin{figure}[H]
    \centering
    \includegraphics[width=0.92\textwidth]{imagens/Tarefa2/5.conf_ospf_area3.png}
    \caption{Configuração do OSPF no Router 5}
    \label{fig:config6}
\end{figure}
\vspace{0.2cm}

\newpage
\subsection{IMPLEMENTING OSPF AUTHENTICATION}
\vspace{0.2cm}

Para garantir a segurança das comunicações entre routers na área OSPF, iremos implementar a autenticação MD5 no OSPF. Para tal, vamos configurar com os seguintes passos:
\vspace{0.2cm}

\begin{enumerate}
  \item \textbf{Definir a chave de autenticação} para o OSPF, utilizando o comando \textit{ip ospf message-digest-key} seguido do número da chave, do tipo de cifra (MD5) e da palavra-passe.
  \item \textbf{Ativar a autenticação} no OSPF, utilizando o comando \textit{area} seguido do número da área OSPF e de \textit{authentication message-digest}.
\end{enumerate}
\vspace{0.2cm}

Para o router 3, as configurações foram realizadas com sucesso, como se pode observar na figura \ref{fig:config7}.

\begin{figure}[H]
    \centering
    \includegraphics[width=0.92\textwidth]{imagens/Tarefa2/6.ospf_auth.png}
    \caption{Configuração da autenticação MD5 no Router 3}
    \label{fig:config7}
\end{figure}

\newpage
\subsection{CONFIGURING OSPF NETWORK TYPES}
\vspace{0.2cm}

Seguidamente, vamos configurar os tipos de rede OSPF ponto a ponto nas nas interfaces com sub-redes /30. Para tal, vamos configurar com os seguintes passos:

\begin{enumerate}
  \item \textbf{Escolher a interface} que será configurada como ponto a ponto, utilizando o comando \textit{interface} seguido do tipo de interface.
  \item \textbf{Definir o tipo de rede} como ponto a ponto, utilizando o comando \textit{ip ospf network point-to-point}.
\end{enumerate}
\vspace{0.2cm}

Para o router 10, as configurações foram realizadas com sucesso, como se pode observar na figura \ref{fig:config8}.
\vspace{0.2cm}

\begin{figure}[H]
    \centering
    \includegraphics[width=0.92\textwidth]{imagens/Tarefa2/7.ospf_network_types.png}
    \caption{Configuração do tipo de rede OSPF ponto a ponto no Router 10}
    \label{fig:config8}
\end{figure}
\vspace{0.2cm}

\newpage
\subsection{CONTROLLING DR/BDR ELECTION}
\vspace{0.2cm}

Por fim, vamos controlar a eleição do Designated Router (DR) e do Backup Designated Router (BDR) nas interfaces OSPF. O R1 será o DR e o R3 será o BDR. Para tal, vamos configurar com os seguintes passos:
\vspace{0.2cm}

\begin{enumerate}
  \item \textbf{Escolher a interface} que será configurada como DR, utilizando o comando \textit{interface} seguido do tipo de interface.
  \item \textbf{Definir a prioridade} do router para a eleição do DR, utilizando o comando \textit{ip ospf priority} seguido do valor pretendido.
\end{enumerate}
\vspace{0.2cm}

Para os routers 1, 3 e 6 as configurações foram realizadas com sucesso, como se pode observar na figura \ref{fig:config9}.

\begin{figure}[h]
  \centering
  \begin{subfigure}{.53\textwidth}
      \centering
      \includegraphics[width=1\linewidth]{imagens/Tarefa2/8.dr_bdr_R1.png}
      \caption{Definir o DR no Router 1}
      \label{fig:router1}
  \end{subfigure}%
  \begin{subfigure}{.4\textwidth}
      \centering
      \includegraphics[width=1\linewidth]{imagens/Tarefa2/8.dr_bdr_R3.png}
      \caption{Definir o BDR no Router 3}
      \label{fig:router3}
  \end{subfigure}%

  \vspace{0.1cm}
  \begin{subfigure}{.53\textwidth}
      \centering
      \includegraphics[width=1\linewidth]{imagens/Tarefa2/8.dr_bdr_R6.png} 
      \caption{Impedir que R6 participe na eleição do DR}
      \label{fig:router6}
  \end{subfigure}
  \caption{Controlar a eleição do DR e BDR nos Routers 1, 3 e 6}
  \label{fig:config9}
\end{figure}
\vspace{0.2cm}

\newpage
\subsection{VERIFICATION AND TROUBLESHOOTING}
\vspace{0.2cm}

Para verificar a configuração do OSPF e a eleição do DR e BDR, vamos utilizar os seguintes comandos:
\vspace{0.2cm}

\begin{itemize}
  \item \textbf{show ip ospf neighbor}: Mostra a lista de vizinhos OSPF com os quais o router estabeleceu adjacências. Tal como se pode observar na figura \ref{fig:config10}.
  \vspace{0.2cm}

  \begin{figure}[H]
    \centering
    \includegraphics[width=0.75\textwidth]{imagens/Tarefa2/9.ospf_neigh.png}
    \caption{Verificar os vizinhos OSPF no Router 1}
    \label{fig:config10}
  \end{figure}
  \vspace{0.2cm}

  \item \textbf{show ip route ospf}: Mostra a tabela de routing OSPF, que inclui as rotas aprendidas através do OSPF. Tal como se pode observar na figura \ref{fig:config11}.
  \vspace{0.2cm}

  \begin{figure}[H]
    \centering
    \includegraphics[width=0.75\textwidth]{imagens/Tarefa2/9.ospf_route.png}
    \caption{Verificar a tabela de routing OSPF no Router 1}
    \label{fig:config11}
  \end{figure}
  \vspace{0.2cm}

  \item \textbf{show ip ospf database}: Mostra a base de dados OSPF, que inclui informações sobre as redes OSPF, os routers vizinhos e os links (LSAs) entre routers. Tal como se pode observar na figura \ref{fig:config12}.
  \vspace{0.2cm}

  \begin{figure}[H]
    \centering
    \includegraphics[width=0.83\textwidth]{imagens/Tarefa2/9.ospf_database.png}
    \caption{Verificar a base de dados OSPF no Router 1}
    \label{fig:config12}
  \end{figure}
  \vspace{0.2cm}

  \item \textbf{show ip ospf interface}: Mostra informações detalhadas sobre as interfaces OSPF, incluindo o estado da interface, o tipo de rede OSPF (como point-to-point ou broadcast) e o router ID. Tal como se pode observar na figura \ref{fig:config13}.
  \vspace{0.2cm}

  \begin{figure}[H]
    \centering
    \includegraphics[width=0.92\textwidth]{imagens/Tarefa2/9.ospf_interface.png}
    \caption{Verificar as interfaces OSPF no Router 1}
    \label{fig:config13}
  \end{figure}
  \vspace{0.2cm}
\end{itemize}

\subsection{REVIEW QUESTIONS}
\vspace{0.2cm}

\begin{enumerate}
  \item \textbf{Why is it important to manually set OSPF router IDs?} 
  \vspace{0.2cm}

  \par Definir manualmente os IDs dos routers no OSPF garante uma rede estável e organizada, prevenindo problemas de convergência. Caso o router ID seja atribuído automaticamente, pode mudar caso o IP de uma interface ativamente configurada seja alterado, causando recalculações de routing e instabilidade. O router ID manual assegura consistência, especialmente em cenários multi-área.
  \vspace{0.2cm}

  \item \textbf{What is the purpose of OSPF authentication, and why is MD5 preferred over clear text?}
  \vspace{0.2cm}

  \par A autenticação OSPF é um mecanismo fundamental que garante a integridade e autenticidade das mensagens trocadas entre routers, protegendo a rede contra ameaças como spoofing e injeções de rotas maliciosas. A utilização do MD5 é preferida em relação ao texto claro, uma vez que oferece uma segurança superior ao criptografar as atualizações de routing, dificultando significativamente a falsificação de mensagens por parte de potenciais atacantes.
  \vspace{0.2cm}

  \item \textbf{In what scenarios would you choose to configure an OSPF network type as point-to-point?}
  \vspace{0.2cm}

  \par O tipo de rede OSPF ponto a ponto é geralmente utilizado em cenários onde existem ligações diretas entre dois routers, sem a presença de switches ou hubs. Este tipo de rede é adequado para ligações dedicadas, como ligações seriais ponto a ponto, túneis VPN ou ligações Ethernet ponto a ponto.
  \vspace{0.2cm}

  \par As principais vantagens de configurar uma rede OSPF como ponto a ponto incluem:
  \begin{itemize}
    \item \textbf{Menor sobrecarga de tráfego}: A comunicação é direta entre os routers, sem a necessidade de enviar pacotes de broadcast para todos os routers na rede.
    \item \textbf{Maior eficiência}: A comunicação é mais eficiente e rápida, uma vez que não é necessário eleger um DR/BDR ou enviar pacotes de hello para todos os routers na rede.
    \item \textbf{Maior segurança}: A comunicação é mais segura, uma vez que é direta entre os routers e não é transmitida para outros dispositivos na rede.
  \end{itemize}

  \item \textbf{Explain the roles of DR and BDR in OSPF, and why we might want to control their election.}
  \vspace{0.2cm}

  \par O Designated Router (DR) coordena a comunicação em redes OSPF multiacesso, enquanto o Backup Designated Router (BDR) assume a função do DR em caso de falha. Controlar suas eleições garante estabilidade e desempenho, assegurando que routers mais capazes desempenhem essas funções.
  \vspace{0.2cm}

  \item \textbf{What happens when a router with a higher priority than the DR or BDR is added?}
  \vspace{0.2cm}

  \par Se um router com prioridade mais alta for adicionado, ele será eleito como DR ou BDR durante a próxima eleição, substituindo o router atual. Isso pode causar instabilidade temporária enquanto a nova configuração é propagada.
  \vspace{0.2cm}

  \item \textbf{On the segment S2 there are adjacencies with status 2WAY, explain the observed behaviour.}
  \vspace{0.2cm}

  \par O estado 2WAY indica que os routers podem trocar pacotes hello, mas não estabeleceram uma adjacência FULL. Isso pode ocorrer em redes multiacesso sem DR/BDR ou em redes ponto a ponto. Problemas de conectividade também podem impedir a transição para o estado FULL.
  \vspace{0.2cm}

  \item \textbf{WHow does the point-to-point network type differ from the other OSPF network types?}
  \vspace{0.2cm}

  \par OEm redes ponto a ponto, a comunicação ocorre diretamente entre dois routers, sem a necessidade de DR/BDR, o que reduz o overhead e aumenta a segurança e eficiência, ao contrário das redes broadcast ou ponto-a-multiponto.
  \vspace{0.2cm}

  \item \textbf{Are hello packets for this network type (P2P) sent as unicast or multicast?}
  \vspace{0.2cm}

  \par Em redes ponto a ponto, os pacotes hello são enviados como unicast, garantindo uma comunicação direta e eficiente entre os dois routers, sem tráfego desnecessário para outros dispositivos.
  \vspace{0.2cm}

  \item \textbf{Does the P2P network type support the DR/BDR election? Why or why not?}
  \vspace{0.2cm}

  \par O tipo de rede ponto a ponto não suporta eleição de DR/BDR, pois a comunicação é direta entre dois routers, tornando desnecessária a coordenação de múltiplos dispositivos.
  \vspace{0.2cm}

  \item \textbf{Activate in R1 and R2 the debug ip ospf events. Change the hello timers on the interface between R1 and 
  R2 to half of the current value. Wait for the dead interval. What was the impact from this change on the 
  topology?}
  \vspace{0.2cm}

  \par Após a ativação do comando \textit{debug ip ospf events} nos routers R1 e R2 e a alteração dos temporizadores hello na interface entre eles para metade do valor atual, utilizando o comando \textit{ip ospf hello-interval 5}, observou-se um impacto significativo na topologia da rede.
  \vspace{0.2cm}

  \par Após reduzir os temporizadores hello entre R1 e R2, a adjacência OSPF foi interrompida, resultando na perda de conectividade e reconfiguração da topologia. Esse impacto mostra a importância dos temporizadores hello e dead para a estabilidade e convergência do OSPF.
  \vspace{0.2cm}

  Como se pode observar na figura \ref{fig:config14}.
  \vspace{0.2cm}

  \begin{figure}[H]
    \centering
    \includegraphics[width=0.92\textwidth]{imagens/Tarefa2/10.hello.png}
    \caption{Alteração dos vizinhos OSPF após a alteração do temporizador hello}
    \label{fig:config14}
  \end{figure}
  \vspace{0.2cm}

  \item \textbf{What is the relation between the hello and dead interval? Revert the hello timer to the previous value.}
  \vspace{0.2cm}

  \par O temporizador dead é quatro vezes o valor do temporizador hello. Se o temporizador hello for alterado, o dead também deve ser ajustado proporcionalmente para garantir a remoção eficiente de adjacências inativas. Após a alteração, é importante reverter o temporizador hello ao valor original (\textit{ip ospf hello-interval 10}) para manter a estabilidade da rede e reestabelecer as adjacências OSPF.
\end{enumerate}
\vspace{0.2cm}

\subsection{LAB TASKS}
\vspace{0.2cm}

Após concluir todas as configurações e responder às perguntas, iremos verificar a conectividade entre os PCs, especificamente entre o PC22 e o PC33, assim como a conectividade entre os routers nas Áreas 0 e 3.
\vspace{0.2cm}

Primeiramente, iremos fazer uma tabela com os endereços IP dos PCs, tal como se pode observar na tabela \ref{tab:ip2}.

\begin{table}[H]
\centering
\begin{tabular}{|c|c|c|c|}
\hline
\textbf{Dispositivo} & \textbf{Endereço IP} & \textbf{Máscara de Sub-rede} & \textbf{Endereço IP de Gateway}\\ \hline
PC11 & 10.7.11.2 & 255.255.255.252 & 10.7.11.1 \\ \hline 
PC22 & 10.5.6.5 & 255.255.255.248 & 10.5.6.1 \\ \hline
PC33 & 10.1.33.2 & 255.255.255.252 & 10.1.33.1 \\ \hline
PC44 & 10.2.8.4 & 255.255.255.248 & 10.2.8.1 \\ \hline
PC55 & 10.10.55.2 & 255.255.255.252 & 10.10.55.2\\ \hline
\end{tabular}
\caption{Endereços IP dos PCs}
\label{tab:ip2}
\end{table}
\vspace{0.2cm}

Seguidamente, vammos atribuir os endereços IP aos PCs, utilizando o comando \textit{ip} seguido do endereço IP e da máscara de sub-rede. Para o PC22, as configurações foram realizadas com sucesso, como se pode observar na figura \ref{fig:config15}.
\vspace{0.2cm}

\begin{figure}[H]
    \centering
    \includegraphics[width=0.5\textwidth]{imagens/Tarefa2/11.pc22_conf.png}
    \caption{Configuração do endereço IP no PC22}
    \label{fig:config15}
\end{figure}
\vspace{0.2cm}

Posteriormente, vamos verificar a conectividade entre os PCs, utilizando o comando \textit{ping} seguido do endereço IP do PC de destino. Para tal, é necessário verificar se o PC de origem consegue enviar pacotes ICMP para o PC de destino e se este consegue responder aos mesmos. Como se pode observar na figura \ref{fig:config16}, a conectividade foi verificada com sucesso entre os PCs 22 e 33.

\begin{figure}[H]
    \centering
    \includegraphics[width=0.75\textwidth]{imagens/Tarefa2/11.ping_pc22_pc33.png}
    \caption{Verificar a conectividade entre os PCs 22 e 33}
    \label{fig:config16}
\end{figure}
\vspace{0.2cm}

\newpage
Por fim, para verificar a conectividade entre os routers nas Áreas 0 e 3, utilizamos o comando \textit{ping} seguido do endereço IP do router de destino. Como se pode observar na figura \ref{fig:config17}, a conectividade foi verificada com sucesso entre router 5 e os routers nas Áreas 0 e 3.

\begin{figure}[H]
    \centering
    \includegraphics[width=0.92\textwidth]{imagens/Tarefa2/11.ping_R5.png}
    \caption{Verificar a conectividade entre o Router 5 e os Routers nas Áreas 0 e 3}
    \label{fig:config17}
\end{figure}

\pagebreak

\section{Tarefa 3 - OSPF AREA 1 AND VIRTUAL LINK CONFIGURATION}
\vspace{0.2cm}

Nesta tarefa, o objetivo é configurar o OSPF na Área 1 e estabelecer um virtual link para conectar a Área 1 ao backbone (Área 0) através da Área 3. Inicialmente, será configurado o OSPF nos routers R7 e R5 na Área 1. Em seguida, um virtual link será criado entre o router R5 (Área 3) e os routers R1 e R3 (Área 0), garantindo a conectividade entre as áreas.

Além disso, abordaremos a importância dos router IDs para a configuração do virtual link, bem como as limitações e erros comuns. Para otimizar a distribuição de rotas, transformaremos a Área 1 numa rede stub no-summary, reduzindo o tráfego de routing. No final, será verificada a conectividade entre todas as áreas, assegurando o correto funcionamento do OSPF na rede.
\vspace{0.2cm}

\subsection{CONFIGURING OSPF IN AREA 1}
\vspace{0.2cm}

Para os routers R7 e R5, vamos configurar o OSPF na Área 1 com os seguintes passos:
\vspace{0.2cm}

\begin{enumerate}
  \item \textbf{Ativar o OSPF} no router, utilizando o comando \textit{router ospf} seguido do número do processo OSPF.
  \item \textbf{Definir o router ID} do router, utilizando o comando \textit{router-id} seguido do endereço IP da loopback.
  \item \textbf{Configurar as interfaces} do router para o OSPF, utilizando o comando \textit{network} seguido do endereço IP da interface, da wildcard mask e da área OSPF.
\end{enumerate}
\vspace{0.2cm}

Para o router 7, as configurações foram realizadas com sucesso, como se pode observar na figura \ref{fig:config18}.

\begin{figure}[H]
    \centering
    \includegraphics[width=0.92\textwidth]{imagens/Tarefa3/12.config_R7_ospf.png}
    \caption{Configuração do OSPF no Router 7}
    \label{fig:config18}
\end{figure}
\vspace{0.2cm}

Para o router 5, as configurações foram realizadas com sucesso, como se pode observar na figura \ref{fig:config19}.
\vspace{0.2cm}

\begin{figure}[H]
    \centering
    \includegraphics[width=0.92\textwidth]{imagens/Tarefa3/12.config_R5_ospf.png}
    \caption{Configuração do OSPF no Router 5}
    \label{fig:config19}
\end{figure}
\vspace{0.2cm}

\subsection{UNDERSTANDING THE NEED FOR A VIRTUAL LINK}
\vspace{0.2cm}

Para garantir a conectividade entre a Área 1 e o backbone (Área 0) em OSPF, quando não existe uma ligação direta entre elas, é necessário criar um virtual link. Como a Área 1 não está conectada diretamente à Área 0, podemos estabelecer uma ligação lógica através da Área 3, configurando o link entre os routers da Área 3 (R5) e os routers da Área 0 (R1 e R3).

Este virtual link permite que as áreas não contíguas troquem informações de routing, assegurando a comunicação entre elas, mesmo sem uma ligação física direta. Assim, o virtual link é uma solução eficiente para conectar áreas que não são adjacentes e não partilham uma área comum, mantendo a conectividade e a continuidade da rede OSPF.
\vspace{0.2cm}

Vamos utilizar os seguintes comandos:
\vspace{0.2cm}

\begin{enumerate}
  \item \textbf{router ospf 1}: Ativa o OSPF no router, utilizando o número do processo OSPF.
  \item \textbf{area 3 virtual-link}: Cria um virtual link entre as áreas OSPF, especificando o endereço IP do router vizinho na área 3.
  \item \textbf{authentication message-digest}: Ativa a autenticação MD5 para o virtual link, garantindo a segurança da comunicação.
  \item \textbf{message-digest-key 1 md5 vlink}: Define a chave de autenticação MD5 para o virtual link, assegurando a integridade das mensagens.
  \item \textbf{authentication-key vlink}: Define a chave de autenticação para o virtual link, garantindo a segurança da comunicação.
\end{enumerate}
\vspace{0.2cm}

\newpage
Para o router 5, as configurações foram realizadas com sucesso, como se pode observar na figura \ref{fig:config20}.
\vspace{0.2cm}

\begin{figure}[H]
    \centering
    \includegraphics[width=0.92\textwidth]{imagens/Tarefa3/13.virtual_link_R5.png}
    \caption{Configuração do virtual link no Router 5}
    \label{fig:config20}
\end{figure}
\vspace{0.2cm}

Para os routers 1 e 3, as configurações foram realizadas com sucesso, como se pode observar na figura \ref{fig:config21}.
\vspace{0.2cm}

\begin{figure}[H]
    \centering
    \includegraphics[width=0.92\textwidth]{imagens/Tarefa3/13.virtual_link_R1.png}
    \caption{Configuração do virtual link nos Routers 1 e 3}
    \label{fig:config21}
\end{figure}
\vspace{0.2cm}

\newpage 
\subsection{OPTIMIZING ROUTE DISTRIBUTION}
\vspace{0.2cm}

Para otimizar a distribuição de rotas na rede, podemos configurar a Área 1 como uma rede stub no-summary. Como a Área 1 está localizada no final da rede, com uma única conexão à rede, essa configuração ajuda a reduzir o tráfego de routing, evitando a propagação de informações de routing desnecessárias para a área. Além disso, ao transformá-la em uma rede stub, a Área 1 não irá propagar informações de routing para outras áreas, o que melhora a eficiência e a segurança da rede OSPF. Essa abordagem resulta em uma rede mais otimizada, com um menor número de rotas e maior controle sobre o tráfego de routing.
\vspace{0.2cm}

Para os routers R7 e R5, vamos configurar a Área 1 como uma rede stub no-summary com os seguintes passos:

\begin{enumerate}
  \item \textbf{router ospf 1}: Ativa o OSPF no router, utilizando o número do processo OSPF.
  \item \textbf{area 1 stub no-summary}: Configura a Área 1 como uma rede stub no-summary, evitando a propagação de informações de routing para outras áreas.
\end{enumerate}
\vspace{0.2cm}

Para o router 5, as configurações foram realizadas com sucesso, como se pode observar na figura \ref{fig:config22}.
\vspace{0.2cm}

\begin{figure}[H]
    \centering
    \includegraphics[width=0.92\textwidth]{imagens/Tarefa3/14.optimizing_route_R5.png}
    \caption{Configuração da Área 1 como uma rede stub no-summary no Router 5}
    \label{fig:config22}
\end{figure}
\vspace{0.2cm}

\newpage
\subsection{VERIFICATION AND TROUBLESHOOTING}
\vspace{0.2cm}

Para verificar a configuração do OSPF e do virtual link, bem como a otimização da distribuição de rotas, vamos utilizar os seguintes comandos:
\vspace{0.2cm}

\begin{itemize}
  \item \textbf{show ip ospf virtual-links}: Mostra informações sobre os links virtuais OSPF, incluindo o estado do link, o endereço IP do router vizinho e a autenticação MD5. Tal como se pode observar na figura \ref{fig:config23}.
  \vspace{0.2cm}

  \begin{figure}[H]
    \centering
    \includegraphics[width=0.75\textwidth]{imagens/Tarefa3/15.virtual_link_test_R5.png}
    \caption{Verificar o virtual link no Router 5}
    \label{fig:config23}
  \end{figure}
  \vspace{0.2cm}

  \item \textbf{show ip ospf neighbor}: Mostra a lista de vizinhos OSPF com os quais o router estabeleceu adjacências. Tal como se pode observar na figura \ref{fig:config24}.
  \vspace{0.2cm}

  \begin{figure}[H]
    \centering
    \includegraphics[width=0.92\textwidth]{imagens/Tarefa3/15.ospf_neigh_R5.png}
    \caption{Verificar os vizinhos OSPF no Router 5}
    \label{fig:config24}
  \end{figure}
  \vspace{0.2cm}

  \item \textbf{show ip route ospf}: Mostra a tabela de routing OSPF, que inclui as rotas aprendidas através do OSPF. Tal como se pode observar na figura \ref{fig:config25}.
  \vspace{0.2cm}

  \begin{figure}[H]
    \centering
    \includegraphics[width=0.75\textwidth]{imagens/Tarefa3/15.ospf_route_R5.png}
    \caption{Verificar a tabela de routing OSPF no Router 5}
    \label{fig:config25}
  \end{figure}
  \vspace{0.2cm}

  \newpage
  \item \textbf{show ip ospf database}: Mostra a base de dados OSPF, que inclui informações sobre as redes OSPF, os routers vizinhos e os links (LSAs) entre routers. Tal como se pode observar na figura \ref{fig:config26}.
  \vspace{0.2cm}

  \begin{figure}[H]
    \centering
    \includegraphics[width=0.75\textwidth]{imagens/Tarefa3/15.ospf_database_R5.png}
    \caption{Verificar a base de dados OSPF no Router 5}
    \label{fig:config26}
  \end{figure}
  \vspace{0.2cm}
\end{itemize}

\newpage
\subsection{REVIEW QUESTIONS}
\vspace{0.2cm}

\begin{enumerate}
  \item \textbf{What is the purpose of an OSPF virtual link?}
  \vspace{0.2cm}

  \par O virtual link OSPF permite interligar áreas OSPF não contíguas, garantindo a comunicação entre elas através de uma ligação lógica, mesmo sem uma conexão física direta. Ele é usado quando duas áreas não compartilham uma área comum, mas precisam trocar informações de routing. Isso é feito estendendo a área backbone (área 0) para incluir áreas não adjacentes, mantendo a conectividade e propagação de rotas.
  \vspace{0.2cm}

  \item \textbf{Why is authentication important on a virtual link?}
  \vspace{0.2cm}

  \par A autenticação é essencial para garantir a segurança da troca de informações de routing e evitar ataques como spoofing e injeção de rotas maliciosas. A autenticação MD5, com uma chave secreta partilhada, assegura que as mensagens OSPF não foram alteradas, evitando a propagação de rotas fraudulentas e protegendo contra ataques man-in-the-middle.
  \vspace{0.2cm}

  \item \textbf{How does a virtual link affect the path that packets take through the network?}
  \vspace{0.2cm}

  \par O virtual link permite que os pacotes atravessem áreas OSPF não diretamente conectadas, ao criar uma ligação lógica entre elas. Isso pode alterar a rota dos pacotes, pois eles podem passar por áreas que não seriam diretamente acessíveis, garantindo a continuidade da comunicação entre as áreas.
  \vspace{0.2cm}

  \item \textbf{What are some limitations or potential issues with using virtual links?}
  \vspace{0.2cm}

  \par Existem várias limitações e potenciais problemas ao utilizar virtual links no OSPF, que devem ser cuidadosamente considerados ao planejar a rede. Alguns dos principais pontos incluem:

  \begin{itemize}
    \item \textbf{Complexidade de Configuração}: A configuração de links virtuais pode ser difícil e exige conhecimento da topologia da rede.
    \item \textbf{Instabilidade}: Links mal configurados podem causar falhas de conectividade e afetar a troca de informações de routing.
    \item \textbf{Segurança}: Links virtuais podem ser vulneráveis a ataques, por isso a autenticação é essencial.
    \item \textbf{Impacto na Convergência}: Problemas de conectividade podem aumentar o tempo de convergência do OSPF.
    \item \textbf{Dependência da Área Backbone}: OA falha no link virtual ou na área 0 pode afetar toda a rede OSPF.
  \end{itemize}
  \vspace{0.2cm}

  \newpage
  \item \textbf{What happens when area 0 becomes partitioned?}
  \vspace{0.2cm}

  \par Quando a área 0 é dividida/partitioned, são definidas duas outras sub-áreas de backbone que agem em separado, onde cada uma irá efetuar o tratamento do encaminhamento das rotas inter-área.
  \vspace{0.2cm}

  \par A criação de novas sub-áreas de backbone não deverá afetar o funcionamento e a relação destas com qualquer outra área na rede, sendo que estas continuarão a ser as áreas de backbone da topologia de rede em questão.
  \vspace{0.2cm}

  \item \textbf{What does the IP address used in the virtual-link statement refer to?}
  \vspace{0.2cm}

  \par O endereço IP refere-se ao Router ID do router vizinho que está na área adjacente. Esse Router ID é utilizado para estabelecer a conexão lógica entre áreas não contíguas.
  \vspace{0.2cm}

  \item \textbf{Must this address be reachable before the virtual-link can be established?}
  \vspace{0.2cm}

  \par Sim, o endereço IP do router vizinho deve ser alcançável para que o virtual link funcione corretamente. Caso contrário, a comunicação entre as áreas não será possível.
  \vspace{0.2cm}

  \item \textbf{How can we optimize the route tables of non-transit areas?}
  \vspace{0.2cm}

  \par Para otimizar as tabelas de routing de áreas não transitórias no OSPF, é possível configurá-las como áreas stub ou áreas stub no-summary. Essas configurações ajudam a melhorar o desempenho da rede, reduzindo a quantidade de informações de routing propagadas e, consequentemente, o tráfego de routing entre as áreas.

  \par Aqui estão as opções para otimizar as tabelas de routing:
  \vspace{0.2cm}

  \begin{itemize}
    \item \textbf{Área Stub}: Restringe a propagação de rotas externas, permitindo apenas a rota padrão.
    \item \textbf{Área Stub No-Summary}: Além das rotas externas, evita a propagação de rotas de resumo.
    \item \textbf{Área Totally Stubby}: Não propaga rotas externas nem de resumo, permitindo apenas as rotas locais e a padrão.
  \end{itemize}
  \vspace{0.2cm}

  \par Estas configurações reduzem o tráfego e as tabelas de routing, melhorando a eficiência da rede.
  \vspace{0.2cm}
  \end{enumerate}

\newpage
\subsection{LAB TASKS}
\vspace{0.2cm}

\par Após concluir todas as configurações e responder às perguntas, o objetivo deste laboratório é verificar a conectividade entre os dispositivos. As tarefas incluem a verificação da conectividade completa OSPF entre todos os routers nas Áreas 0, 1 e 3, a monitorização do caminho de um pacote de R7 para R2, documentando cada salto, e a avaliação da conectividade ICMP entre os PCs PC11 <-> PC22 e PC11 <-> PC33.

\par Primeiramente, vamos verificar a conectividade entre os routers nas Áreas 0, 1 e 3, utilizando o comando \textit{ping} seguido do endereço IP do router de destino. Para tal, é necessário verificar se o router de origem consegue enviar pacotes ICMP para o router de destino e se este consegue responder aos mesmos. Como se pode observar na figura \ref{fig:config27}, a conectividade foi verificada com sucesso entre os routers nas Áreas 0, 1 e 3 (R7 -> R2 e R7 -> R6).

\begin{figure}[H]
    \centering
    \includegraphics[width=0.6\textwidth]{imagens/Tarefa3/16.ping_R7_R2_R6.png}
    \caption{Verificar a conectividade entre os Routers nas Áreas 0, 1 e 3}
    \label{fig:config27}
\end{figure}
\vspace{0.1cm}

\par Posteriormente, vamos monitorizar o caminho de um pacote de R7 para R2, utilizando o comando \textit{traceroute} seguido do endereço IP do router de destino. Este comando permite documentar cada salto que o pacote faz ao longo do caminho, mostrando os routers intermediários que o pacote atravessa. Como se pode observar na figura \ref{fig:config28}, o caminho do pacote de R7 para R2 passa pelos routers R5 e R1.

\begin{figure}[H]
    \centering
    \includegraphics[width=0.50\textwidth]{imagens/Tarefa3/16.traceroute_R7_R2_R6.png}
    \caption{Monitorizar o caminho de um pacote de R7 para R2 e para }
    \label{fig:config28}
\end{figure}
\vspace{0.2cm}

\newpage
\par Por fim, vamos verificar a conectividade ICMP entre os PCs PC11 e PC22, assim como entre os PCs PC11 e PC33, utilizando o comando \textit{ping} seguido do endereço IP do PC de destino. Como se pode observar na figura \ref{fig:config29}, a conectividade foi verificada com sucesso entre os PCs PC11 e PC22, assim como entre os PCs PC11 e PC33.
\vspace{0.2cm}

\begin{figure}[H]
    \centering
    \includegraphics[width=0.82\textwidth]{imagens/Tarefa3/16.ping_PC11_PC22_PC33.png}
    \caption{Verificar a conectividade entre os PCs PC11 e PC22, PC11 e PC33}
    \label{fig:config29}
\end{figure} 
\pagebreak

\section{Tarefa 4 - OSPF AREA 2 AND EXTERNAL ROUTE REDISTRIBUTION}
\vspace{0.2cm}

Nesta tarefa, o objetivo é configurar o OSPF na Área 2 e integrar rotas externas através da redistribuição de RIP. Inicialmente, serão configurados os routers R2, R8 e R9 para suportar o OSPF na Área 2. Em seguida, será estabelecida a redistribuição de rotas RIP nas interfaces específicas dos routers R10 e R9 para conectar a Área 2 ao backbone (Área 0), permitindo a troca de rotas entre OSPF e RIP.

Será implementada a redistribuição de rotas do RIP para o OSPF, assegurando a conectividade entre redes externas e a topologia OSPF. Para melhorar o controlo das LSAs (Link-State Advertisements), a Área 2 será configurada como uma NSSA (Not-So-Stubby Area), limitando a propagação de LSAs externas desnecessárias. Adicionalmente, os custos das rotas redistribuídas serão ajustados para otimizar o encaminhamento do tráfego.

Por fim, serão utilizados comandos de verificação para validar a configuração e resolver possíveis problemas, garantindo o correto funcionamento do protocolo OSPF e a integração das rotas externas.
\vspace{0.2cm}

\subsection{CONFIGURING OSPF IN AREA 2}
\vspace{0.2cm}

Para os routers R9, R8 e R2, vamos configurar o OSPF na Área 2 com os seguintes passos:
\vspace{0.2cm}

\begin{enumerate}
  \item \textbf{Ativar o OSPF} no router, utilizando o comando \textit{router ospf} seguido do número do processo OSPF.
  \item \textbf{Definir o router ID} do router, utilizando o comando \textit{router-id} seguido do endereço IP da loopback.
  \item \textbf{Configurar as interfaces} do router para o OSPF, utilizando o comando \textit{network} seguido do endereço IP da interface, da wildcard mask e da área OSPF.
\end{enumerate}
\vspace{0.2cm}

Para o router 8, as configurações foram realizadas com sucesso, como se pode observar na figura \ref{fig:config30}:
\vspace{0.2cm}

\begin{figure}[H]
  \centering
  \includegraphics[width=0.70\textwidth]{imagens/Tarefa4/17.config_R8.png}
  \caption{Configuração do OSPF no Router 8}
  \label{fig:config30}
\end{figure}
\vspace{0.2cm}

\subsection{CONFIGURING RIP}
\vspace{0.2cm}

Para os routers R9 e R10, vamos configurar o RIP nas interfaces específicas com os seguintes passos:
\vspace{0.2cm}

\begin{enumerate}
  \item \textbf{Ativar o RIP} no router, utilizando o comando \textit{router rip}.
  \item \textbf{version 2} para ativar o RIP versão 2.
  \item \textbf{Configurar as interfaces} do router para o RIP, utilizando o comando \textit{network} seguido do endereço IP da interface.
  \item \textbf{no auto-summary} para desativar a sumarização automática de rotas.
\end{enumerate}
\vspace{0.2cm}

Para o router 9, as configurações foram realizadas com sucesso, como se pode observar na figura \ref{fig:config31}:
\vspace{0.2cm}

\begin{figure}[H]
  \centering
  \includegraphics[width=0.70\textwidth]{imagens/Tarefa4/18.config_R9.png}
  \caption{Configuração do RIP no Router 9}
  \label{fig:config31}
\end{figure}
\vspace{0.2cm}

\subsection{REDISTRIBUTING RIP INTO OSPF AND OSPF TO RIP}
\vspace{0.2cm}

Para o router R9, vamos redistribuir as rotas RIP no OSPF e ajustar os custos das rotas redistribuídas com os seguintes passos:
\vspace{0.2cm}

\begin{enumerate}
  \item \textbf{router ospf 1}: Ativa o OSPF no router, utilizando o número do processo OSPF.
  \item \textbf{redistribute rip subnets}: Redistribui as rotas RIP no OSPF, incluindo as sub-redes.
  \item \textbf{router rip}: Ativa o RIP no router.
  \item \textbf{redistribute ospf 1 metric 1}: Redistribui as rotas OSPF no RIP, ajustando o custo para 1.
\end{enumerate}
\vspace{0.2cm}

\newpage
Para o router 9, as configurações foram realizadas com sucesso, como se pode observar na figura \ref{fig:config32}:
\vspace{0.2cm}

\begin{figure}[H]
  \centering
  \includegraphics[width=0.70\textwidth]{imagens/Tarefa4/19.redistribute_R9.png}
  \caption{Redistribuição de rotas RIP no OSPF e OSPF no RIP no Router 9}
  \label{fig:config32}
\end{figure}
\vspace{0.2cm}

\subsection{CONTROLLING LSA PROPAGATION}
\vspace{0.2cm}

Para os routers R9, R8 e R2, vamos configurar a Área 2 como uma NSSA (Not-So-Stubby Area) para controlar a propagação de LSAs externas com os seguintes passos:
\vspace{0.2cm}

\begin{enumerate}
  \item \textbf{router ospf 1}: Ativa o OSPF no router, utilizando o número do processo OSPF.
  \item \textbf{area 2 nssa}: Configura a Área 2 como uma NSSA, limitando a propagação de LSAs externas.
\end{enumerate}
\vspace{0.2cm}

Para o router 8, as configurações foram realizadas com sucesso, como se pode observar na figura \ref{fig:config33}:
\vspace{0.2cm}

\begin{figure}[H]
  \centering
  \includegraphics[width=0.70\textwidth]{imagens/Tarefa4/20.nssa_R8.png}
  \caption{Configuração da Área 2 como uma NSSA no Router 8}
  \label{fig:config33}
\end{figure}
\vspace{0.2cm}

\newpage
\subsection{ADJUSTING OSPF METRICS FOR EXTERNAL ROUTES}
\vspace{0.2cm}

Para o router R9, vamos ajustar os custos das rotas OSPF para otimizar o encaminhamento do tráfego com os seguintes passos:
\vspace{0.2cm}

\begin{enumerate}
  \item \textbf{router ospf 1}: Ativa o OSPF no router, utilizando o número do processo OSPF.
  \item \textbf{redistribute rip subnets metric-type 1 metric 1000}: Ajusta o custo das rotas redistribuídas do RIP para 1000.
\end{enumerate}
\vspace{0.2cm}

Para o router 9, as configurações foram realizadas com sucesso, como se pode observar na figura \ref{fig:config34}:
\vspace{0.2cm}

\begin{figure}[H]
  \centering
  \includegraphics[width=0.7\textwidth]{imagens/Tarefa4/21.adjust_R9.png}
  \caption{Ajustar os custos das rotas OSPF no Router 9}
  \label{fig:config34}
\end{figure}
\vspace{0.2cm}

\subsection{VERIFICATION AND TROUBLESHOOTING}
\vspace{0.2cm}

Para verificar a configuração do OSPF, RIP e redistribuição de rotas, bem como a configuração da Área 2 como uma NSSA e o ajuste dos custos das rotas OSPF, vamos utilizar os seguintes comandos:
\vspace{0.2cm}

\begin{itemize}
  \item \textbf{show ip ospf neighbor}: Mostra a lista de vizinhos OSPF com os quais o router estabeleceu adjacências. Tal como se pode observar na figura \ref{fig:config35}.
  \vspace{0.2cm}

  \begin{figure}[H]
    \centering
    \includegraphics[width=0.92\textwidth]{imagens/Tarefa4/22.ospf_neigh_R9.png}
    \caption{Verificar os vizinhos OSPF no Router 9}
    \label{fig:config35}
  \end{figure}
  \vspace{0.2cm}

  \newpage
  \item \textbf{show ip route}: Mostra a tabela de routing, que inclui as rotas aprendidas através do OSPF e do RIP. Tal como se pode observar na figura \ref{fig:config36}.
  \vspace{0.2cm}

  \begin{figure}[H]
    \centering
    \includegraphics[width=0.92\textwidth]{imagens/Tarefa4/22.route_R9.png}
    \caption{Verificar a tabela de routing no Router 9}
    \label{fig:config36}
  \end{figure}
  \vspace{0.2cm}

  \newpage
  \item \textbf{show ip ospf database nssa-external}: Mostra a base de dados OSPF para LSAs externas na NSSA, incluindo informações sobre as rotas redistribuídas. Tal como se pode observar na figura \ref{fig:config37}.
  \vspace{0.2cm}

  \begin{figure}[H]
    \centering
    \includegraphics[width=0.7\textwidth]{imagens/Tarefa4/22.ospf_database_R9.png}
    \caption{Verificar a base de dados OSPF na NSSA no Router 9}
    \label{fig:config37}
  \end{figure}
  \vspace{0.2cm}

  \newpage
  \item \textbf{show ip protocols}: Mostra informações sobre os protocolos de routing configurados, incluindo OSPF e RIP. Tal como se pode observar na figura \ref{fig:config38}.
  \vspace{0.2cm}

  \begin{figure}[H]
    \centering
    \includegraphics[width=0.80\textwidth]{imagens/Tarefa4/22.ip_protocols_R9.png}
    \caption{Verificar os protocolos de routing no Router 9}
    \label{fig:config38}
  \end{figure}
  \vspace{0.2cm}
\end{itemize}

\newpage
\subsection{REVIEW QUESTIONS}
\vspace{0.2cm}

\begin{enumerate}
  \item \textbf{What is the purpose of redistributing routes between routing protocols?}
  \vspace{0.2cm}

  \par A redistribuição de rotas entre protocolos de routing permite a troca de informações entre diferentes domínios de routing, como OSPF e RIP. Este processo é essencial em situações como:
  \vspace{0.2cm}

  \begin{itemize}
    \item \textbf{Integração de redes heterogéneas}: Por exemplo, após uma fusão de empresas, onde cada uma utiliza um protocolo de routing diferente.
    \item \textbf{Migração entre protocolos}: Durante uma transição gradual de um protocolo para outro, permitindo que as redes coexistam temporariamente.
  \end{itemize}
  \vspace{0.2cm}

  Sem redistribuição, estas redes podem ficar isoladas, comprometendo a conectividade e a eficiência global. No entanto, a redistribuição deve ser configurada cuidadosamente para evitar problemas como loops de routing e inconsistências de métricas.
  \vspace{0.2cm}

  \item \textbf{How does configuring an area as NSSA affect LSA propagation?}
  \vspace{0.2cm}

  \par Configurar uma área como NSSA (Not-So-Stubby Area) limita a propagação de LSAs externas, permitindo:
  \vspace{0.2cm}

  \begin{itemize}
    \item \textbf{Controlo de LSAs externas}: Apenas LSAs tipo 7 são permitidas, que são convertidas para LSAs tipo 5 no backbone OSPF, evitando a inundação de rotas externas para outras áreas.
    \item \textbf{Redução de tráfego de routing}: Restringe rotas externas desnecessárias, o que melhora o desempenho em redes grandes.
  \end{itemize}
  \vspace{0.2cm}

  Isto é particularmente útil em redes OSPF de grande escala, onde as áreas NSSA mantêm a escalabilidade ao controlar a injeção de rotas externas sem comprometer a conectividade.
  \vspace{0.2cm}

  \newpage
  \item \textbf{Why might we want to adjust metrics for redistributed routes?}
  \vspace{0.2cm}

  \par Ajustar as métricas para rotas redistribuídas é essencial para:
  \vspace{0.2cm}

  \begin{itemize}
    \item \textbf{Evitar loops de routing}: Garantindo que um protocolo não volta a anunciar as suas próprias rotas redistribuídas. Por exemplo, ao redistribuir rotas de RIP para OSPF, as métricas podem ser configuradas para evitar que as rotas retornem ao RIP.
    \item \textbf{Otimizar o tráfego}: Permite ajustar os custos de rotas para refletir a preferência de encaminhamento, direcionando o tráfego por caminhos mais rápidos ou mais fiáveis.
    \item \textbf{Evitar rotas assimétricas}: Métricas inconsistentes podem levar a situações em que o tráfego segue por um caminho, mas o retorno ocorre por outro, causando latências ou falhas inesperadas.
  \end{itemize}
  \vspace{0.2cm}

  Este ajuste é frequentemente realizado através de comandos como metric ou metric-type nos dispositivos de routing.
  \vspace{0.2cm}

  \item \textbf{What are the potential risks or challenges of route redistribution?}
  \vspace{0.2cm}

  \par A redistribuição de rotas entre protocolos de routing apresenta vários riscos que devem ser mitigados com planeamento adequado:
  \vspace{0.2cm}

  \begin{itemize}
    \item \textbf{Risco de Loops de Rotas}:Pode ocorrer se as rotas redistribuídas forem anunciadas de volta ao protocolo original. Este problema pode ser evitado com políticas de redistribuição e uso de tags para identificar rotas.
    \item \textbf{Convergência Lenta}: Redes redistribuídas podem demorar mais tempo a convergir, afetando a estabilidade em caso de falha de ligação.
    \item \textbf{Inconsistência de Métricas}: Diferenças entre métricas de protocolos como RIP e OSPF podem levar a rotas ineficientes.
    \item \textbf{Risco de Injeção de Rotas Maliciosas}: Rotas indesejadas ou maliciosas podem ser injetadas durante o processo, comprometendo a segurança. Políticas de redistribuição e filtros de rotas ajudam a mitigar este risco.
    \item \textbf{Complexidade de Configuração}:Configurações incorretas podem causar interrupções ou loops. É fundamental conhecer bem a topologia da rede antes de redistribuir rotas.
  \end{itemize}
  \vspace{0.2cm}

  \newpage
  \item \textbf{What different options are there for converting RIP external route costs on the OSPF?}
  \vspace{0.2cm}

  \par Ao redistribuir rotas do RIP para OSPF, os custos das rotas podem ser ajustados de várias formas:
  \vspace{0.2cm}

  \begin{itemize}
    \item \textbf{Redistribuição com Métricas Fixas}: Atribui um custo fixo para todas as rotas redistribuídas, garantindo consistência. Por exemplo, redistribute rip metric 20.
    \item \textbf{Redistribuição com Métricas Dinâmicas}: Ajusta os custos com base em parâmetros dinâmicos, como largura de banda ou atraso.
    \item \textbf{Redistribuição com Métricas Relativas}: Configura as métricas proporcionalmente às métricas RIP originais, refletindo a diferença entre os dois protocolos.
    \item \textbf{Redistribuição com Métricas Personalizadas}: Define métricas adaptadas às necessidades específicas da rede, como custos diferenciados para rotas de backup.
    \item \textbf{Redistribuição com Métricas Padrão ou Estáticas}: Utiliza métricas baseadas em valores padrão ou configuradas manualmente, como largura de banda ou atraso mínimo.
  \end{itemize}
  \vspace{0.2cm}
  \end{enumerate}

\newpage
\subsection{LAB TASKS}
\vspace{0.2cm}

\par Após concluir todas as configurações e responder às perguntas, o objetivo deste laboratório é verificar a conectividade entre os dispositivos.  As tarefas incluem validar a redistribuição de rotas RIP no OSPF, monitorizar o percurso de um pacote de R1 para a rede 10.9.10.0/30, documentando cada salto, e garantir a conectividade ICMP entre os dispositivos, como R10 (interface loopback0) e todas as interfaces loopback dos outros routers, bem como entre os PCs na topologia, incluindo PC55 e os restantes PCs.
\vspace{0.2cm}

\par Primeiramente, vamos verificar a redistribuição de rotas RIP no OSPF, utilizando o comando \textit{show ip route} para verificar as rotas aprendidas através do OSPF e do RIP. Isto já foi feito anteriormente, como se pode observar na figura \ref{fig:config36}.
\vspace{0.2cm}

\par Posteriormente, vamos monitorizar o caminho de um pacote de R1 para a rede 10.9.10.0/30. Para tal, vamos utilizar o comando \textit{traceroute} seguido do endereço IP da rede de destino. Este comando permite documentar cada salto que o pacote faz ao longo do caminho, mostrando os routers intermediários que o pacote atravessa. Como se pode observar na figura \ref{fig:config39}, o caminho do pacote de R1 para a rede 10.9.10.0/30 passa pelos routers R2 e R9.
\vspace{0.2cm}

\begin{figure}[H]
  \centering
  \includegraphics[width=0.82\textwidth]{imagens/Tarefa4/23.traceroute_R1_10.9.10.1.png}
  \caption{Monitorizar o caminho de um pacote de R1 para a rede 10.9.10.0/30}
  \label{fig:config39}
\end{figure}
\vspace{0.2cm}

\newpage
\par Seguidamente, vamos garantir a conectividade ICMP entre os dispositivos, como R10 (interface loopback0) e todas as interfaces loopback dos outros routers. Para tal, vamos utilizar o comando \textit{ping} seguido do endereço IP do dispositivo de destino, bem como \textit{source loopback0}. Como se pode observar na figura \ref{fig:config40}, a conectividade foi verificada com sucesso entre os dispositivos.
\vspace{0.2cm}

\begin{figure}[H]
  \centering
  \includegraphics[width=0.82\textwidth]{imagens/Tarefa4/23.ping_R10_loopback0.png}
  \caption{Verificar a conectividade entre R10 (interface loopback0) e as interfaces loopback dos outros routers}
  \label{fig:config40}
\end{figure}
\vspace{0.2cm}

\newpage
\par Por fim, vamos garantir a conectividade ICMP entre os PCs na topologia, incluindo PC55 e os restantes PCs. Para tal, vamos utilizar o comando \textit{ping} seguido do endereço IP do PC de destino. Como se pode observar na figura \ref{fig:config41}, a conectividade foi verificada com sucesso entre os PCs na topologia.
\vspace{0.2cm}

\begin{figure}[H]
  \centering
  \includegraphics[width=0.82\textwidth]{imagens/Tarefa4/23.ping_PC55_PC11_PC22_PC33.png}
  \caption{Verificar a conectividade entre PC55 e os restantes PCs}
  \label{fig:config41}
\end{figure}
\pagebreak

\section{Tarefa 5 - ADVANCED OSPF CONFIGURATIONS AND MAINTENANCE SCENARIO}
\vspace{0.2cm}

Nesta tarefa, o objetivo é implementar configurações avançadas no OSPF e simular cenários de manutenção para analisar o comportamento do protocolo durante alterações na rede. A configuração irá incluir a otimização da tabela de rotas através de sumarização, a simulação de cenários de manutenção e a validação detalhada das alterações efetuadas.

Primeiramente, será implementada a sumarização de rotas no router R5, consolidando as rotas que este anuncia no OSPF. Este processo reduzirá o tamanho da tabela de rotas, aumentando a eficiência do protocolo e facilitando a gestão da rede.

De seguida, será simulada uma manutenção no link entre os routers R1 e R3, aumentando o custo OSPF para esse link. Esta simulação permitirá desviar o tráfego para rotas alternativas, analisando como o OSPF ajusta dinamicamente as rotas face a mudanças na topologia.

Por fim, serão utilizados comandos de verificação para validar as configurações aplicadas e analisar o comportamento da rede.

\subsection{ROUTE SUMMARIZATION}
\vspace{0.2cm}

Para o router R5, vamos configurar a sumarização de rotas no OSPF com os seguintes passos:
\vspace{0.2cm}

\begin{enumerate}
  \item \textbf{router ospf 1}: Ativa o OSPF no router, utilizando o número do processo OSPF.
  \item \textbf{area 1 range}: Configura a sumarização de rotas para a área OSPF, especificando o endereço IP de início, a wildcard mask e o custo.
\end{enumerate}
\vspace{0.2cm}

Para o router 5, as configurações foram realizadas com sucesso, como se pode observar na figura \ref{fig:config42}:
\vspace{0.2cm}

\begin{figure}[H]
  \centering
  \includegraphics[width=0.92\textwidth]{imagens/Tarefa5/24.summarization_R5.png}
  \caption{Configuração da Sumarização de Rotas no Router 5}
  \label{fig:config42}
\end{figure}
\vspace{0.2cm}

\newpage
\par Para verificar a configuração da sumarização de rotas, podemos utilizar o comando \textit{show ip route} no router R7 para verificar a tabela de routing e confirmar que as rotas foram sumarizadas com sucesso. Também podemos utilizar o comando \textit{show ip ospf database summary}
para verificar a base de dados OSPF e confirmar que a sumarização foi aplicada corretamente, como se pode observar na figura \ref{fig:config42}.
\vspace{0.2cm}

\begin{figure}[H]
  \centering
  \includegraphics[width=0.92\textwidth]{imagens/Tarefa5/24.show_ip_route_R7.png}
  \caption{Verificar a Tabela de Routing no Router 7}
  \label{fig:config42}
\end{figure}
\vspace{0.2cm}

\par A sumarização não está a funcionar porque a Área 2 é uma NSSA, onde as rotas externas (Tipo 7) não podem ser sumarizadas diretamente com o comando \textit{area 2 range}. Para que a sumarização funcione corretamente, é necessário garantir que as se utilize o comando \textit{summary-address} para permitir a sumarização de rotas externas na Área 2 ou alterar o tipo de área para uma área normal (não NSSA).

\subsection{SIMULATING A MAINTENANCE SCENARIO}
\vspace{0.2cm}

Para simular uma manutenção no link entre os routers R1 e R3, vamos aumentar o custo OSPF para esse link com os seguintes passos:
\vspace{0.2cm}

\begin{enumerate}
  \item \textbf{interface G1/0}: Acede à interface do link a ser modificado.
  \item \textbf{ip ospf cost 1000}: Aumenta o custo OSPF para 1000, simbolizando uma falha ou degradação do link.
\end{enumerate}
\vspace{0.2cm}

Para o router 1, as configurações foram realizadas com sucesso, como se pode observar na figura \ref{fig:config43}:
\vspace{0.2cm}

\begin{figure}[H]
  \centering
  \includegraphics[width=0.92\textwidth]{imagens/Tarefa5/25.maintenance_R1.png}
  \caption{Simulação de Manutenção no Link entre R1 e R3 no Router 1}
  \label{fig:config43}
\end{figure}

\subsection{VERIFICATION AND ANALYSIS}
\vspace{0.2cm}

Para verificar a configuração da sumarização de rotas e a simulação de manutenção, bem como analisar o comportamento do OSPF, vamos utilizar os seguintes comandos:
\vspace{0.2cm}

\begin{itemize}
  \item \textbf{show ip route}: Mostra a tabela de routing, que inclui as rotas aprendidas através do OSPF e do RIP. Tal como se pode observar na figura \ref{fig:config42}.
  \vspace{0.2cm}

  \item \textbf{show ip ospf database summary}: Mostra a base de dados OSPF para LSAs de sumarização, incluindo informações sobre as rotas sumarizadas. Tal como se pode observar na figura \ref{fig:config42}.
  \vspace{0.2cm}

  \newpage
  \item \textbf{trace <destination-ip> }: Permite monitorizar o caminho de um pacote para o destino, mostrando os routers intermediários que o pacote atravessa. Tal como se pode observar na figura \ref{fig:config44}.
  \vspace{0.2cm}

  \begin{figure}[H]
    \centering
    \includegraphics[width=0.92\textwidth]{imagens/Tarefa5/25.trace_R7_R10.png}
    \caption{Monitorizar o caminho de um pacote de R7 para R10}
    \label{fig:config44}
  \end{figure}

  \newpage
  \item \textbf{show ip ospf interface}: Mostra informações detalhadas sobre as interfaces OSPF, incluindo custos e estados. Tal como se pode observar na figura \ref{fig:config45}.
  \vspace{0.2cm}

  \begin{figure}[H]
    \centering
    \includegraphics[width=0.80\textwidth]{imagens/Tarefa5/25.ospf_interface_R7.png}
    \caption{Verificar as Interfaces OSPF no Router 7}
    \label{fig:config45}
  \end{figure}
  \vspace{0.2cm}
\end{itemize}

\subsection{REVIEW QUESTIONS}
\vspace{0.2cm}

\begin{enumerate}
  \item \textbf{How does filtering summary LSAs affect the routing table and network visibility?}
  \vspace{0.2cm}

  Os LSAs de sumarização podem afetar a tabela de routing e a visibilidade da rede de várias formas:
  \vspace{0.2cm}

  \begin{itemize}
    \item \textbf{Redução da Tabela de Routing}: A sumarização de rotas reduz o número de entradas na tabela de routing, consolidando rotas semelhantes numa única entrada. Isto melhora a eficiência e escalabilidade do protocolo OSPF.
    \item \textbf{Ocultação de Detalhes}: Ao sumarizar rotas, detalhes específicos de sub-redes individuais podem ser ocultados, tornando a rede menos visível em termos de topologia e endereçamento.
    \item \textbf{Otimização do Encaminhamento}: A sumarização pode otimizar o encaminhamento do tráfego, direcionando-o por caminhos mais eficientes e reduzindo a latência.
    \item \textbf{Impacto na Convergência}: Alterações na sumarização podem afetar a convergência do protocolo OSPF, especialmente durante eventos de falha ou manutenção. Em certos casos, uma sumarização incorreta pode resultar em routing subótimo ou falhas temporárias na rede.
  \end{itemize}
  \vspace{0.2cm}

  \item \textbf{What are the benefits and potential drawbacks of route summarization?}
  \vspace{0.2cm}

  A sumarização de rotas oferece vários benefícios e desvantagens que devem ser considerados ao projetar uma rede OSPF:
  \vspace{0.2cm}

  \begin{itemize}
    \item \textbf{Benefícios}:
    \begin{itemize}
      \item \textbf{Redução da Tabela de Routing}: Diminui o tamanho da tabela de routing, melhorando a eficiência e escalabilidade do protocolo.
      \item \textbf{Otimização do Encaminhamento}: Direciona o tráfego por caminhos mais eficientes, reduzindo a latência e melhorando o desempenho.
      \item \textbf{Consolidação de Rotas}: Agrupa rotas semelhantes numa única entrada, simplificando a gestão e manutenção da rede.
      \item \textbf{Segurança e Privacidade}: Oculta detalhes específicos da rede, melhorando a segurança e privacidade dos endereços IP.
    \end{itemize}

    \newpage
    \item \textbf{Desvantagens}:
    \begin{itemize}
      \item \textbf{Perda de Visibilidade}: A sumarização pode ocultar detalhes importantes da rede, dificultando a resolução de problemas e a análise da topologia.
      \item \textbf{Impacto na Convergência}: Alterações na sumarização podem afetar a convergência do protocolo OSPF, causando routing subótimo ou falhas temporárias.
      \item \textbf{Complexidade de Configuração}: A sumarização requer uma configuração cuidadosa para evitar problemas como loops de routing ou falhas de conectividade.
      \item \textbf{Risco de Erros}: Erros na sumarização podem resultar em routing ineficiente, falhas de conectividade ou problemas de segurança.
    \end{itemize}
  \end{itemize}


  \item \textbf{How does changing OSPF interface costs impact routing decisions?}
  \vspace{0.2cm}

  A alteração dos custos das interfaces OSPF pode ter um impacto significativo nas decisões de routing, afetando o caminho preferencial para o tráfego. Alguns efeitos incluem:
  \vspace{0.2cm}

  \begin{itemize}
    \item \textbf{Redirecionamento de Tráfego}: Aumentar o custo de uma interface pode desencorajar o OSPF de utilizar esse caminho, direcionando o tráfego para rotas alternativas com custos mais baixos.
    \item \textbf{Redundância e Resiliência}: Ao aumentar os custos de interfaces redundantes, o OSPF pode favorecer caminhos mais fiáveis, melhorando a resiliência da rede e garantindo maior disponibilidade.
    \item \textbf{Otimização de Encaminhamento}: Ajustar os custos para refletir a largura de banda, atraso ou fiabilidade das interfaces pode otimizar o encaminhamento do tráfego, melhorando o desempenho geral da rede.
    \item \textbf{Impacto na Convergência}: Alterações nos custos das interfaces podem afetar a convergência do protocolo OSPF, causando reajustes na tabela de routing e potenciais falhas temporárias na rede.
  \end{itemize}
  \vspace{0.2cm}

  \newpage
  \item \textbf{In what scenarios might you want to manipulate OSPF costs in a production network?}
  \vspace{0.2cm}

  A manipulação dos custos OSPF é útil em várias situações para otimizar o encaminhamento do tráfego e melhorar a resiliência da rede:
  \vspace{0.2cm}

  \begin{itemize}
    \item \textbf{Redundância e Resiliência}: Ajustar os custos de interfaces redundantes para favorecer caminhos mais fiáveis, garantindo que o tráfego siga por rotas de backup em caso de falha.
    \item \textbf{Otimização de Encaminhamento}: Ajustar os custos com base na largura de banda, atraso ou fiabilidade das interfaces para direcionar o tráfego por caminhos mais eficientes.
    \item \textbf{Balanceamento de Carga}: Ajustar os custos para distribuir o tráfego de forma equilibrada entre várias rotas, evitando congestionamentos e melhorando o desempenho.
    \item \textbf{Priorização de Tráfego}: Ajustar os custos para priorizar determinados tipos de tráfego, como VoIP ou vídeo, garantindo uma qualidade de serviço adequada.
    \item \textbf{Manutenção e Testes}: Temporariamente aumentar os custos de interfaces durante manutenção ou testes para desviar o tráfego para rotas alternativas e evitar interrupções.
    \item \textbf{Segurança e Privacidade}: Ajustar os custos para ocultar rotas específicas ou limitar o acesso a determinadas partes da rede, melhorando a segurança e privacidade.
    \item \textbf{Resolução de Problemas}: Ajustar os custos para isolar problemas de conectividade, identificar rotas ineficientes ou testar novas configurações antes de implementá-las em produção.
    \item \textbf{Adaptação a Alterações na Rede}: Ajustar os custos para refletir alterações na topologia da rede, como a adição de novos links ou a remoção de rotas obsoletas.
  \end{itemize}
\end{enumerate}
\vspace{0.2cm}

\subsection{LAB TASKS}
\vspace{0.2cm}

\par Após concluir as configurações e responder às perguntas, o objetivo deste laboratório é verificar a conectividade e avaliar o impacto das modificações realizadas na rede. Será comparada a tabela de encaminhamento no R9 antes e depois da implementação da sumarização de rotas no OSPF, monitorizando também o percurso dos pacotes de R7 a R10 e documentando qualquer alteração no caminho. A conectividade ICMP entre os dispositivos será verificada, e os resultados dos comandos de verificação serão capturados para garantir que a rede mantém o seu desempenho. O laboratório inclui ainda uma análise sobre como a sumarização de rotas e o cenário de manutenção afetam o comportamento da rede, com o objetivo de garantir a conectividade e a eficiência do encaminhamento.
\vspace{0.2cm}

\par Primeiramente, vamos verificar a tabela de encaminhamento no R9 antes e depois da implementação da sumarização de rotas no OSPF. Para tal, vamos utilizar o comando \textit{show ip route} para verificar as rotas aprendidas através do OSPF e do RIP. Como se pode observar na figura \ref{fig:config46}, a sumarização não reduziu o número de entradas na tabela de routing.
\vspace{0.2cm}

\begin{figure}[h]
  \centering
  \begin{subfigure}{.51\textwidth}
      \centering
      \includegraphics[width=1\linewidth]{imagens/Tarefa5/25.before_show_ip_route_R9.png}
      \caption{Antes da Sumarização de Rotas}
  \end{subfigure}%
  \begin{subfigure}{.49\textwidth}
      \centering
      \includegraphics[width=.95\linewidth]{imagens/Tarefa5/25.after_show_ip_route_R9.png}
      \caption{Depois da Sumarização de Rotas}
  \end{subfigure}
  \caption{Verificar a Tabela de Routing no Router 9}
  \label{fig:config46}
\end{figure}
\vspace{0.2cm}

\newpage
\par Posteriormente, vamos monitorizar o caminho dos pacotes de R7 a R10, utilizando o comando \textit{tracerout} seguido do endereço IP de destino. Este comando permite documentar cada salto que o pacote faz ao longo do caminho, mostrando os routers intermediários que o pacote atravessa. Não se observaram alterações no caminho dos pacotes, ficou igual ao que se pode observar na figura \ref{fig:config44}.
\vspace{0.2cm}

\par A sumarização de rotas e o cenário de manutenção tiveram um impacto significativo no comportamento da rede, embora algumas expectativas não se tenham concretizado. No caso da sumarização de rotas implementada no Router 5, verificou-se que não houve redução no número de entradas na tabela de routing do R9, indicando que a configuração não foi aplicada corretamente ou que as sub-redes incluídas na sumarização não foram adequadamente ajustadas. Este problema comprometeu o objetivo de simplificar a tabela de encaminhamento e otimizar os recursos da rede. Para resolver esta questão, é necessário rever a configuração da sumarização, assegurando que as rotas são sumarizadas corretamente e alinhadas com as necessidades da rede.

\par Já a simulação de manutenção no link entre os routers R1 e R3, através do aumento do custo OSPF para esse enlace, não resultou em alterações no percurso dos pacotes de R7 a R10. O comando \textit{traceroute} confirmou que o caminho dos pacotes permaneceu inalterado. Apesar disso, a rede manteve a conectividade esperada, demonstrando a estabilidade do protocolo OSPF. Estes resultados reforçam a importância de configurações precisas para que ferramentas como a sumarização de rotas e a manipulação de custos OSPF atinjam o seu pleno potencial na otimização e resiliência da rede.

\pagebreak

\section{Tarefa 6 - OSPF OPTIMIZATION AND TROUBLESHOOTING}
\vspace{0.2cm}

No capítulo anterior, exploramos configurações avançadas do OSPF, abordando a sumarização de rotas para otimizar a tabela de roteamento e simulando cenários de manutenção para analisar o impacto no comportamento da rede. Essas ações proporcionaram insights sobre como gerenciar redes OSPF em situações que requerem alta resiliência e eficiência.

Neste capítulo, avançaremos para técnicas de otimização de tempos de convergência, engenharia de tráfego dinâmica e análise detalhada de cenários de falhas. Este enfoque permitirá explorar soluções para melhorar a capacidade da rede de responder a mudanças de topologia, garantindo um desempenho confiável e previsível.

\subsection{OPTIMIZING OSPF CONVERGENCE}
\vspace{0.2cm}

Otimizar o tempo de convergência do OSPF é essencial para redes que exigem alta disponibilidade e recuperação rápida de falhas. Uma técnica eficiente é configurar pacotes \textbf{"fast hello"}, reduzindo os tempos de detecção de falhas.
\vspace{0.2cm}

\textbf{Configuração:}
Nos routers R1 e R2, aplicamos a seguinte configuração nas interfaces que os conectam (GigabitEthernet0/0 em ambos):

\begin{figure}[H]
  \centering
  \includegraphics[width=0.82\textwidth]{imagens/Tarefa6/27.ospf_convergence_R2.png}
  \caption{Configuração de técnicas para melhorar tempo de convergência do OSPF nas interfaces que conectam R1 e R2}
  \label{fig:config42}
\end{figure}


\begin{itemize}
  \item \textbf{ip ospf dead-interval minimal}: Configura o OSPF para utilizar um intervalo de inatividade mínimo, ajustando-se ao envio de pacotes hello em intervalos menores.
  \item \textbf{hello-multiplier 4}: Define que quatro pacotes hello serão enviados por segundo, acelerando a detecção de falhas.
\end{itemize}

Essa configuração reduz significativamente o tempo de convergência, permitindo que os vizinhos OSPF detectem a indisponibilidade de um link em milissegundos.


\pagebreak
\subsection{OSPF TRAFFIC ENGINEERING}
\vspace{0.2cm}

A engenharia de tráfego permite influenciar a seleção de rotas, ajustando custos dinamicamente com base no estado do link. Utilizamos ~\textbf{IP SLA} para monitorizar a latência, rastreamento para avaliar o estado do link e EEM para alterar automaticamente os custos do OSPF. \par \vspace{0.1cm}

\vspace{0.8cm}
\textbf{Configuração em R3}
\begin{enumerate}
  \item \textbf{Monitorização do link com IP SLA} \par \vspace{0.2cm}

    \textbf{IP SLA:} Ferramenta que permite monitorizar a qualidade do link em termos de latência, perda de pacotes e conectividade. Usamos o IP SLA para monitorizar em tempo real o estado de um link crítico, enviando pacotes ICMP (pings) para verificar se o link está ativo ou se está degradado além de um limite aceitável.

    \begin{verbatim}
      ip sla 1
      icmp-echo 10.3.4.[R4-g1/0-ipaddress] source-interface GigabitEthernet0/0
      frequency 5
      threshold 50
      ip sla schedule 1 life forever start-time now
    \end{verbatim}

    \begin{itemize}
      \item \textbf{icmp-echo}: Monitoriza a conectividade com o endereço do R4, simulando pacotes ICMP.
      \item \textbf{frequency 5}: Define uma frequência de monitorizamento a cada 5 segundos.
      \item \textbf{threshold 50}: Estabelece um limite de latência em milissegundos.
    \end{itemize}

    \vspace{0.4cm}
  \item \textbf{Rastreamento do estado do link} \par

    \begin{verbatim}
      track 1 ip sla 1 state
    \end{verbatim}

    \textbf{Track (Rastreamento):} Mecanismo que avalia o estado de um objeto, como um IP SLA, para tomar decisões com base no seu status (ativo ou inativo). Vinculamos o rastreamento ao IP SLA para que possamos monitorizar automaticamente o estado do link e acionar mudanças no custo OSPF em caso de falha ou degradação. \par \vspace{0.2cm}

    O rastreamento vincula o estado do IP SLA a um objeto de monitorização, permitindo que ações automáticas sejam tomadas com base no estado do link (ativo ou inativo).

    \pagebreak
  \item \textbf{Automação com EEM para ajustar custos do OSPF}
    \par \vspace{0.2cm}

    \textbf{EEM (Embedded Event Manager):} ferramenta que permite executar scripts ou comandos automaticamente quando ocorre um evento específico na rede. Configuramos o EEM para alterar o custo do OSPF quando o estado do rastreamento indica uma falha ou para restaurar o custo quando o link retorna ao normal.

    \vspace{0.2cm}
    \begin{itemize}
      \item \textbf{Em caso de falha:}
      \begin{verbatim}
        event manager applet CHANGE_OSPF_COST
        event track 1 state down
        action 1.0 cli command "enable"
        action 2.0 cli command "configure terminal"
        action 3.0 cli command "interface GigabitEthernet0/0"
        action 4.0 cli command "ip ospf cost 1000"
        action 5.0 cli command "end"
      \end{verbatim}

      Quando o Track detecta que o link monitorado pelo IP SLA está inativo (estado down), o EEM altera o custo OSPF da interface para 1000. \par
      Este aumento do custo força o OSPF a evitar o uso desse link como caminho preferencial, redirecionando o tráfego para rotas alternativas.

      \vspace{0.8cm}

      \item \textbf{Em caso de recuperação:}
      \begin{verbatim}
        event manager applet RESTORE_OSPF_COST
        event track 1 state up
        action 1.0 cli command "enable"
        action 2.0 cli command "configure terminal"
        action 3.0 cli command "interface GigabitEthernet0/0"
        action 4.0 cli command "ip ospf cost 10"
        action 5.0 cli command "end"
      \end{verbatim}

      Quando o Track detecta que o link voltou a ficar ativo (estado up), o EEM restaura o custo OSPF da interface para 10. \par
      Isso permite que o OSPF volte a considerar o link como preferencial, se for a rota de menor custo.
  \end{itemize}
\end{enumerate}

\pagebreak
Portanto:
\begin{itemize}
  \item O IP SLA monitora a qualidade do link.
  \item O Track avalia o estado do monitoramento e detecta mudanças (ativo/inativo).
  \item O EEM age sobre essas mudanças, ajustando automaticamente o custo OSPF para influenciar as decisões de roteamento.
\end{itemize}

Essa abordagem dinâmica permite que o OSPF adapte os caminhos do tráfego de maneira automática e eficiente, garantindo o melhor desempenho possível mesmo em cenários adversos.

\vspace{0.8cm}

\begin{figure}[h]
  \centering
  \begin{subfigure}{.8\textwidth}
    \centering
      \includegraphics[width=1\linewidth]{imagens/Tarefa6/28.ospf_traffic_engineering_R3_1.png}
  \end{subfigure}
  \begin{subfigure}{.8\textwidth}
    \centering
      \includegraphics[width=1\linewidth]{imagens/Tarefa6/28.ospf_traffic_engineering_R3_2.png}
  \end{subfigure}
  \caption{Configuração de técnicas de OSPF Traffic Engineering no router R3}
  \label{fig:config43}
\end{figure}
\vspace{0.2cm}


\pagebreak
\subsection{FAILURE SCENARIO ANALYSIS}
\vspace{0.2cm}

A análise de falhas fornece uma compreensão detalhada da resiliência do OSPF e do impacto das mudanças de topologia. Os seguintes cenários foram simulados:

\begin{enumerate}
  \item \textbf{Link Congestion (Congestionamento no Link):} \par \vspace{0.2cm}
    \textbf{Objetivo:} Analisar o impacto da latência no tráfego e a capacidade de adaptação do OSPF. \par \vspace{0.2cm}
    
    \textbf{Passos:}
      \begin{enumerate}
        \item Criar congestionamento artificial
          \begin{itemize}
            \item Executar um trace de R7 para o endereço IP de PC55:
              \begin{figure}[H]
                \centering
                \includegraphics[width=0.6\textwidth]{imagens/Tarefa6/29.link_congestion_1.png}
                \caption{Traceroute de R7 para PC55}
                \label{fig:config43}
              \end{figure}
            
            \pagebreak
            \item Ao mesmo tempo, fazer um ping entre R3 e R4:
              \begin{figure}[H]
                \centering
                \includegraphics[width=0.7\textwidth]{imagens/Tarefa6/29.link_congestion_2.png}
                \caption{Ping de R3 para R4}
                \label{fig:config44}
              \end{figure}   
          \end{itemize}
        
        \item Analisar os resultados \par \vspace{0.2cm}
        
        - \textbf{Quando o IP SLA está no estado up (sem congestionamento percebido):} \par

        O tráfego segue a rota original, que passa pelo link R3–R4 e atinge o destino sem redirecionamento. \par
        O traceroute neste estado mostra que o tráfego flui pelo caminho mais curto calculado pelo OSPF, priorizando o link entre R3 e R4. \par \vspace{0.6cm}

        
        - \textbf{Quando o IP SLA muda para o estado down (congestionamento detectado):} \par
        
        O IP SLA deveria sinalizar a falha ao rastrear atrasos excessivos no link. \par
        No entanto, pelo que foi observado, o OSPF não alterou a rota; o tráfego continua passando pelo mesmo link congestionado (R3–R4), evidenciado pelos tempos de resposta altos e a presença do *.
        
        \pagebreak
      \end{enumerate}

  \item \textbf{Link Failure (Falha no link):} \par \vspace{0.2cm}
      \textbf{Objetivo:} Simular falha numa interface \par \vspace{0.2cm}
      \textbf{Passos:}
        \begin{enumerate}
          \item Desligar (shutdown) à interface que liga R3 a R4 (GigabitEthernet0/0)
          \begin{figure}[H]
            \centering
            \includegraphics[width=0.8\textwidth]{imagens/Tarefa6/30.link_failure_1.png}
            \caption{Desligar interface entre R3 e R4}
            \label{fig:config45}
          \end{figure}   
          
          \item Analisar os resultados
          \begin{figure}[H]
            \centering
            \includegraphics[width=0.8\textwidth]{imagens/Tarefa6/30.link_failure_2.png}
            \caption{Tabela OSPF Neighbors após shutdown interface}
            \label{fig:config46}
          \end{figure}
        \end{enumerate}

        Observa-se os seguintes impactos:
        \begin{itemize}
          \item \textbf{Perda de adjacência com R4:} Como a interface foi desativada, o R3 e o R4 não podem mais trocar informações de roteamento.
          Isso afeta todas as rotas que dependiam do link entre os dois roteadores. Essas rotas são removidas das tabelas de roteamento em ambos os dispositivos.
          \item \textbf{Redirecionamento de Tráfego:} O OSPF recalcula as melhores rotas com base nas métricas disponíveis.
          O tráfego que antes usava o link direto entre R3 e R4 agora será redirecionado por rotas alternativas.
          \item \textbf{Convergência:} OSPF levou cerca de 40 segundos para remover o vizinho da tabela, como indicado pelo Dead Time padrão (40 segundos).
        \end{itemize} \par \vspace{0.8cm}
  
        A simulação de falha pela desativação da interface GigabitEthernet0/0 no R3 confirma o funcionamento esperado do OSPF: detecta a falha, remove a adjacência e recalcula as rotas.
  \pagebreak
  \item \textbf{Router Failure (Falha no Router):} \par \vspace{0.2cm}
  \textbf{Objetivo:} Simular falha num router \par \vspace{0.2cm}
  \textbf{Passos:}
        \begin{enumerate}
          \item Simular falha no router R3 desligando-o
          \begin{figure}[H]
            \centering
            \includegraphics[width=0.8\textwidth]{imagens/Tarefa6/31.router_failure_1.png}
            \caption{Desligar router R3}
            \label{fig:config47}
          \end{figure}
            Desligamos apenas o dispositivo no ambiente.
            \vspace{0.4cm}

          \pagebreak
          \item Observar o impacto na rede
            \begin{enumerate}
              \item \textbf{OSPF Neighbors} \par \vspace{0.2cm}
                É mostrado os OSPF neighbors nos dispositivos R1 e R4 (adjacentes a R3) antes e depois de R3 ser desligando
                \begin{figure}[h]
                  \centering
                  \begin{subfigure}{.8\textwidth}
                    \centering
                    \includegraphics[width=0.75\textwidth]{imagens/Tarefa6/31.router_failure_2.png}
                  \end{subfigure}
                  \begin{subfigure}{.8\textwidth}
                    \centering
                    \includegraphics[width=0.75\textwidth]{imagens/Tarefa6/31.router_failure_3.png}
                  \end{subfigure}
                  \caption{OSPF neighbors nos routers R1 e R4}
                  \label{fig:config48}
                \end{figure}

                É possível observar que os dispositivos vizinhos detectam a falha ao deixarem de receber pacotes Hello do R3. \par
                O estado OSPF passa de FULL para DOWN para todas as adjacências envolvendo o R3.

              \pagebreak
              \item \textbf{OSPF Routing Table} \par \vspace{0.2cm}
                Verificam-se as tabelas de roteamento OSPF nos routers R1 e R4
                \vspace{0.4cm}
                \begin{figure}[h]
                  \centering
                  \subfloat{{\includegraphics[width=0.4\textwidth]{imagens/Tarefa6/31.router_failure_4.png} }}%
                  \qquad
                  \subfloat{{\includegraphics[width=0.4\textwidth]{imagens/Tarefa6/31.router_failure_5.png} }}%
                  \caption{Tabelas de roteamento OSPF nos routers R1 e R4}%
                  \label{fig:example}
                \end{figure} \par \vspace{0.4cm}

                As rotas aprendidas através do R3 são removidas da base de dados OSPF. \par
                Os dispositivos da rede recalculam os seus caminhos, contornando o R3.
              
              \pagebreak
              \item \textbf{Testes de Conectividade} \par \vspace{0.2cm}
                São usados os comandos \textbf{ping} e \textbf{show ip route} para testar a conectividade. É feito um envio desde o router R6 para o router R4, antes e depois de ser desligado o router R3, para verificar se existem alterações no funcionamento. \par \vspace{0.4cm}
                
                \begin{figure}[H]
                  \centering
                  \includegraphics[width=0.6\textwidth]{imagens/Tarefa6/31.router_failure_6.png}
                  \caption{Teste de conectividade com falha no R3}
                \end{figure}

                Analisando os resultados podemos observar:
                \begin{itemize}
                  \item \textbf{Preferência de Rota Antes da Falha:} Mesmo com o R3 operacional, o R6 utilizou a rota via R1 em vez de diretamente pelo R3.
                  Isso ocorre devido à configuração de IP SLA no R3, que dinamicamente aumenta o custo OSPF para 1000 ao detectar alterações de estado na ligação. Esta configuração torna o caminho pelo R3 menos preferível em relação à rota alternativa via R1 e R2, que possui um custo cumulativo menor.
                  
                  \item \textbf{Impacto da Falha do R3:} Quando o R3 foi desligado, o R6 perdeu a adjacência OSPF com ele, mas a tabela de rotas do R6 não sofreu alterações significativas porque já estava a utilizar a rota alternativa via R1.
                  O tempo de convergência foi mínimo, pois a rota via R1 era a preferida antes mesmo da falha.

                  \item \textbf{Alteração no Tempo de Resposta:} O ligeiro aumento no tempo de resposta (74 ms para 89 ms) pode ser explicado por mudanças subtis na rede devido à perda de LSAs originadas pelo R3, o que pode ter causado ajustes nos cálculos de OSPF e no balanceamento de carga no caminho.
                \end{itemize}              
            \end{enumerate}

        \end{enumerate}
  

\end{enumerate}




\pagebreak
\subsection{Review Questions}

\begin{enumerate}
  \item \textbf{How do fast hello packets affect OSPF convergence?}:
  \vspace{0.2cm}

  \par No OSPF, a configuração de pacotes Hello rápidos pode ter um impacto significativo no tempo de convergência, ou seja, o tempo necessário para a rede se ajustar a alterações, como falhas de links ou mudanças na topologia.
  \vspace{0.2cm}

  \par Os pacotes Hello rápidos afetam a convergência do OSPF das seguintes formas:
  \vspace{0.2cm}

  \begin{itemize}
    \item \textbf{Detecção mais rápida de falhas:}  Configurar intervalos Hello mais curtos (por exemplo, 1 segundo) permite detetar mais rapidamente falhas nos vizinhos OSPF. O OSPF utiliza pacotes Hello para verificar se os vizinhos estão ativos. Se um router não receber pacotes Hello dentro do intervalo Dead Interval especificado, considera o vizinho como "inativo" e inicia o processo de reconvergência. Intervalos Hello mais curtos também reduzem proporcionalmente o Dead Interval, o que acelera a deteção de falhas e o processo de reconvergência para escolher novas rotas.
    \vspace{0.2cm}

    \item \textbf{Menor tempo para a transição de estado de vizinhança:} Com intervalos Hello mais curtos, o processo de troca de informações e formação de vizinhança torna-se mais rápido, o que acelera a convergência após alterações na topologia ou reinicializações de routers.
    \vspace{0.2cm}

    \item \textbf{Convergência mais rápida em ambientes dinâmicos:} Redes com mudanças frequentes (como as exigidas neste tipo de projeto) beneficiam de intervalos Hello mais curtos, pois isto acelera a deteção de falhas e reduz o tempo para a rede se ajustar a mudanças na topologia.
  \end{itemize}
  \vspace{0.2cm}

  \item \textbf{What are the potential benefits and risks of OSPF traffic engineering?}
  \vspace{0.2cm}

  \par A engenharia de tráfego (TE) com OSPF apresenta benefícios importantes, mas também implica alguns riscos que devem ser considerados.
  \vspace{0.2cm}

  \textbf{Benefícios:}
  \vspace{0.2cm}
  \begin{itemize}
      \item \textbf{Otimização de Recursos:} A engenharia de tráfego permite utilizar de forma mais eficiente a capacidade e topologia da rede, direcionando o tráfego por caminhos que atendam a requisitos específicos de largura de banda e latência.
      \item \textbf{Flexibilidade de Encaminhamento:} Com a engenharia de tráfego, é possível definir caminhos explícitos para fluxos de tráfego, evitando congestionamentos e melhorando o desempenho global da rede.
  \end{itemize}
  \vspace{0.2cm}

  \newpage
  \textbf{Riscos}:
  \vspace{0.2cm}
  \begin{itemize}
      \item \textbf{Maior complexidade de configuração:} A implementação da engenharia de tráfego com OSPF pode aumentar a complexidade da configuração e manutenção da rede, exigindo planeamento detalhado e conhecimentos especializados.
      \item \textbf{Sobrecarga de Processamento:} A engenharia de tráfego pode introduzir uma sobrecarga adicional nos routers devido ao processamento de informações adicionais sobre o estado dos links e cálculos de caminhos, o que pode afetar o desempenho de dispositivos com recursos limitados.
  \end{itemize}
  \vspace{0.2cm}
  
  \item \textbf{Describe the process you would follow to troubleshoot an OSPF neighbour relationship issue.}
  \vspace{0.2cm}

  Para resolver um problema numa relação de vizinhança OSPF, os seguintes passos podem ser seguidos:
  \vspace{0.2cm}

    \begin{enumerate}
    
        \item \textbf{Verificar o estado da vizinhança}: Usar o comando \textit{show ip ospf neighbor} para confirmar o estado da vizinhança.
        
        \item \textbf{Validar configurações básicas}: Verificar se os IDs dos routers são únicos, se estão na mesma área OSPF e se as interfaces estão configuradas corretamente (mesmo segmento, incluídas no processo OSPF, etc.).
        
        \item \textbf{Testar a conectividade}: Usar os comandos \textit{ping} e \textit{traceroute} para confirmar conectividade IP entre os vizinhos.
        
        \item \textbf{Confirmar a autenticação}: Validar se a autenticação OSPF está ativa e se as chaves coincidem nos routers vizinhos.
        
        \item \textbf{Analisar temporizadores}: Certificar que os intervalos Hello e Dead são iguais em ambos os lados, usando o comando \textit{show ip ospf interface}.
        
        \item \textbf{Verificar o tipo de rede}: Confirmar que o tipo de rede OSPF é compatível entre os routers, com o comando \textit{show ip ospf interface}.
        
        \item \textbf{Analisar logs e mensagens de debug}: Usar \textit{debug ip ospf adjacency} para identificar problemas como incompatibilidades de MTU ou IDs de router.
        
        \item \textbf{Verificar MTU}: Confirmar que as interfaces têm valores de MTU compatíveis, pois discrepâncias podem impedir que atinjam o estado "Full".
        
        \item \textbf{Avaliar configurações avançadas}: Identificar possíveis filtros de LSA ou problemas na eleição de DR/BDR em redes broadcast.
        
        \item \textbf{Testar soluções}: Após aplicar alterações, usar comandos como \textit{show ip ospf neighbor} para confirmar se o problema foi resolvido.
        
    \end{enumerate}
    \vspace{0.2cm}
  
  \item \textbf{How can you use IP SLA with OSPF to create more resilient networks?}
  \vspace{0.2cm}

  O IP SLA (IP Service Level Agreement), combinado com OSPF, pode aumentar a resiliência da rede ao permitir monitorização ativa de desempenho e decisões de encaminhamento dinâmicas.
  \vspace{0.2cm}

  \textbf{Benefícios para a Resiliência:}
  \vspace{0.2cm}
  \begin{enumerate}
      \item \textbf{Failover Automático:}
      Caso o IP SLA detete falhas (e.g., destino inatingível), o OSPF pode redirecionar o tráfego para caminhos alternativos. Este mecanismo reduz tempos de indisponibilidade e melhora a disponibilidade da rede.
      \item \textbf{Encaminhamento Baseado na Qualidade:}
      OSPF pode evitar caminhos congestionados ou com alta latência, preferindo os que apresentam melhor desempenho. Ideal para aplicações sensíveis como voz e vídeo.
      \item \textbf{Melhor Gestão de Links de Backup:}
      O IP SLA permite ativar ou desativar links secundários conforme necessário, garantindo que são utilizados apenas em caso de falhas nos principais.
  \end{enumerate}
  \vspace{0.2cm}
\end{enumerate}

\pagebreak

\chapter{Conclusões}
\vspace{0.2cm}

\par Com a realização deste trabalho prático, foi possível consolidar de forma significativa os conhecimentos teóricos e práticos sobre os protocolos de encaminhamento RIP e OSPF. A configuração de redes OSPF multi-área, a implementação de autenticação, virtual links e a gestão de propagação e resumo de rotas foram aplicadas com sucesso, demonstrando a complexidade e a flexibilidade deste protocolo.

\par Além disso, práticas como a redistribuição de rotas entre OSPF e outros protocolos, a otimização do desempenho e da convergência, e a resolução de problemas comuns enriqueceram o processo de aprendizagem, destacando a importância de uma abordagem metodológica na configuração e gestão de redes. A uniformidade na aplicação do ID do processo OSPF, a escolha criteriosa dos IDs dos routers e o tipo de rede adequado para links ponto a ponto reforçaram os conceitos essenciais para um design eficiente.

\par Este projeto não apenas permitiu a aplicação prática dos conhecimentos adquiridos nas aulas de Redes de Internet, como também preparou para desafios reais na configuração e gestão de redes. Conclui-se, assim, que este trabalho foi fundamental para o desenvolvimento de competências avançadas em protocolos de encaminhamento, promovendo uma compreensão profunda e prática dos conceitos aprendidos.
\pagebreak

\begin{thebibliography}{4} % 100 is a random guess of the total number of
  %references
  \bibitem{Slides} Documentos de apoio da UC e material fornecido pelo docente

  \bibitem{Cisco} Configure Authentication in Open Shortest Path First:\par 
  \url{https://www.cisco.com/c/en/us/support/docs/ip/open-shortest-path-first-ospf/13697-25.html#:~:text=MD5%20authentication%20provides%20higher%20security,a%20non%2Ddecreasing%20sequence%20number}

  \bibitem{OSPF} Understand OSPF Neighbor States:\par 
  \url{https://www.cisco.com/c/en/us/support/docs/ip/open-shortest-path-first-ospf/13685-13.html}

  \bibitem{OSPF} The DR and BDR Roles:\par 
  \url{https://www.packetcoders.io/ospf-the-dr-and-bdr-roles/#:~:text=Within%20OSPF%2C%20the%20role%20of,same%2C%20multiaccess%20broadcast%20network%20segment}

  \bibitem{OSPF} LSA Types Explained:\par 
  \url{https://networklessons.com/ospf/ospf-lsa-types-explained}

  \bibitem{OSPF} Configure the OSPF Not-So-Stubby Area (NSSA):\par 
  \url{https://www.cisco.com/c/en/us/support/docs/ip/open-shortest-path-first-ospf/6208-nssa.html}
    
  \bibitem{OSPF} Administração
  de Redes,  2019/20: \par
  \url{https://www.dcc.fc.up.pt/~rprior/1920/AR/slides/04%20-%20OSPF.pdf}
\end{thebibliography}

\mainmatter
\end{document}